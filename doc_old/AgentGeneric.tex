\def\behaviour{\mathrm{behaviour}}
\def\act{\mathrm{act}}
\def\react{\mathrm{react}}
\def\state{\mathrm{state}}
\def\action{\mathrm{action}}
\def\msg{\mathrm{message}}



\section{Agents}

There is no generally accepted definition of an ``agent'' \cite{MAS-Survey},
therefore it would be defined to suit the needs of this work.

\medskip

\noindent
The \emph{negotiating agents} are isolated proactive computational entities,
capable of sending and receiving messages.
The \emph{isolation} denotes that agents' internal states are protected
from outside access. Messaging is the only way an agent can be interacted with.
The \emph{pro-activity} implies a capacity of acting asynchronously,
with no ``external'' cause.

\medskip

An agent is defined by it's behaviour --- the combination of its
\emph{proactive} and \emph{reactive} (message handling) functions.

\begin{flalign*}
  &\behaviour = \left< \behaviour_\act, \behaviour_\react \right>\\
  &\behaviour_\act   : \state \mapsto \action \\
  &\behaviour_\react : \state \times \msg \mapsto \action
\end{flalign*}

Therefore two agents with same behaviour functions should be considered two instances
of the same agent. In must be noted, that all the diference in the behaviour of
two instances is produced by the differences in the states
(both agent's internal state and the environment's one).

\medskip

The agents, participating in a negotiation, are considered \emph{heterogeneous}
(to any degree: from complete heterogeneity to homogeneity).

In order to generalize some agents behaviour, agent \emph{roles} are introduced.
A role describes whom or what an agents represents in the negotiation and
defines \emph{behaviour archetype} --- the rules to build
agent's \emph{behaviour functions}, given some \emph{role-specific} knowledge.

The agents must use some \emph{communication protocol}, that ensures
understanding between agents of the same or different roles.

% \red{I'm completely stuck: i've got nothing else to write here.}
