\documentclass[../header]{subfiles}

\providecommand{\rootdir}{..}

\def\domain{\mathrm{domain}}

\def\domain{\mathrm{domain}}
\def\pop{\mathrm{pop}}

\def\behaviour{\mathrm{behaviour}}
\def\act{\mathrm{act}}
\def\react{\mathrm{react}}
\def\state{\mathrm{state}}
\def\action{\mathrm{action}}
\def\msg{\mathrm{message}}

\def\coh{\mathrm{coh}}
\def\cohi{\mathrm{\widetilde{coh}}}
\def\rel{\mathrm{rel}}
\def\fold{\mathit{fold}\,}

\def\restrC{\accentset{C}{\xi}}
\def\restrT{\accentset{T}{\xi}}
\def\restrS{\accentset{S}{\xi}^p_i}
\def\restrW{\accentset{W}{\xi}^p_i}

\def\ctx{\mathit{ctx}}
\def\codom{\mathrm{codomain}}
\def\maybe{\mathrm{Maybe\,}}


\subsection{Solving UCSP with Agents Negotiation}
\label{sec:MAS-UCSP}

Agents negotiation was already used to solve University Class Scheduling Problem,
while the problem itself wasn't the researchers main objective \cite{MAS-UCSP}.
Mainly, the authors have developed a negotiation where the agents try to better
understand each others' needs and preferences, that permits to select proposals,
that are more likely to be accepted by the counterparts/

The UCSP problem itself is defined rather simply, but it allows the professors,
represented by agents in the negotiations, to establish their
own constraints, called User Belief --- \underline{personalize} agents' behaviour.

\begin{displayquote}
  The System obtains the User Belief from each agent, then generates
  numbers of different proposals through the Proposal Generator
  according to agents' various preferences and enters the
  Negotiation Process phase. During the negotiation process,
  each agent will select the acceptable proposal from all the
  proposals. The agents then submit this proposal and make
  efforts to reach a consensus with other agents. If there is a
  consensus among all the agents on a specific proposal, this
  proposal becomes the final agreement proposal and the
  Negotiation ends. If the attempt to reach a consensus has been
  unsuccessful, the System goes to Argumentation process where
  agents create arguments according to their own beliefs, and
  proceed with Argumentation with other agents. After the
  Argumentation phase comes the Belief Evolution, where
  beliefs are updated and intensified according to the outcome of
  Argumentation. The system then goes back to the start and
  proceeds with the next negotiation, re-evaluating proposals
  until a general consensus is reached.
  $\dots$ The agent belief includes its preffered time and the school timetable.
  \cite{MAS-UCSP}
\end{displayquote}


\noindent
Some notable features:
\begin{itemize}
  \item In order to select which agent should concede during the negotiation,
    the lowest risk value --- how much utility an agent
    will lose during the concession, is used.
  \item Agents \emph{priority} is based assigned according to the teacher's
    teaching years and job title.
  \item The purpose of the argumentation system is to provide
    agents with the ability to evaluate and generate arguments.
  \item During the argumentation process, the agents pass on the
    argument to each other. The loosing agents  get persuaded,
    and their beliefs must be evolved.
\end{itemize}



\end{document}
