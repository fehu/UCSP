\section{Proposed Solution}
\green{Explicar la solución de manera informal, y después ...}
\medskip


\def\ctx{\mathit{ctx}}
\def\codom{\mathrm{codomain}}
\def\maybe{\mathrm{Maybe~}}


\noindent
The solution to \emph{university classes scheduling problem} (UCSP), proposed
in this thesis, is based on a \emph{negotiation} within a \emph{Multiagent System},
properly discussed in section \ref{sec:NegotiatingAgents}.
Decision processes of the negotiating agents are \emph{coherence-based}
that are discussed in section \ref{sec:coherence}.
Here follows a short summary:
\begin{itemize}
  \item Each agent represents a \emph{professor}, a \emph{group} or a \emph{classroom},
    that are to be put schedule together.
  \item Each agent has internal states, accessible only to it, that hold
    agent's contexts. The contexts contain agent-specific information and constraints.
    Some context variables may/must change during a negotiation.
  \item Agents can interact only using communication: by sending and receiving messages.
  \item Agents can process messages and execute another action at the same time.
  \item Agent's goal is to find a configuration of classes, that complies with
    all the restrictions, contained in the contexts.
  \item Agent's decision process is based on the coherence between the existing
      \emph{classes proposals}, called \emph{classes pool}, and its contexts.
      The classes proposals and one of the contexts, specifically its constraints,
      are used to form valid configurations of classes, called \emph{solution candidates}.
      The candidates are then propagated through the rest of the contexts in the
      defined order. At each context the candidate's coherence is assessed using
      context-specific constraints and the established value is compared against
      a context-specific threshold in order to be propagated further.

      The final decision is based on a set of \emph{acceptable candidates}, that
      have complied successfully with all the contexts' restrictions. It should
      choose whether to:
      \begin{itemize}
        \item add new classes proposals (and remove some old), in case no
          acceptable candidate was found;
        \item select the candidate with maximum coherence as the solution and
          request acceptance confirmation from the ``neighboring'' agents.
      \end{itemize}

  \item The contexts are divided into
    \begin{itemize}[leftmargin=2cm]
      \item[Internal:] define different aspects of agent-specific \emph{internal coherence}
      with solution candidates:
        \begin{enumerate*}[1)]
          \item capabilities,
          \item time consistency,
          \item obligations,
          \item preferences.
        \end{enumerate*}
      \item[External:] communicates with the ``neighboring'' agents and fetches
        their opinions about the candidates, thus cooperating towards the
        \emph{common goal}.
    \end{itemize}
 \end{itemize}

External context combines the \emph{internal coherences} of a candidate,
assessed by the agents, involved in the candidate's underlying classes.
The combination is done, giving no priority to any of the agents,
thus creating an important property:
\begin{displayquote}
  Given a candidate $\tilde{c}$, each agent $g_i, p_i, r_i$, mentioned by
  candidate's underlying classes $\{c_k\}$,
  yields the same coherence assessment for the candidate.
\end{displayquote}

\begin{align}
  \label{eq:coh-fun-independ}
  \begin{aligned}
    &\forall \tilde{c} \sim \{c_k\} \\
    &\forall c_i \sim \left< \dots, g_i, p_i, r_i, \dots \right> \\
    &\forall c_j \sim \left< \dots, g_j, p_j, r_j, \dots \right>
  \end{aligned}
& \implies
  \begin{aligned}
   \coh[g_i](\tilde c) &= \coh[g_j](\tilde c) = \\
   = \coh[p_i](\tilde c) &= \coh[p_j](\tilde c) = \\
   = \coh[r_i](\tilde c) &= \coh[r_j](\tilde c)
  \end{aligned}
\end{align}


\subsection{Solution Formalization}
%%%%%%%%%%%%%%%%%%%%%%%%%%%%%%%%%%%%%%%%%%%%%%%%%%%%%%%%%%%%%%%%%%%%%%%%%%%%%%%%
\green{... y después de manera formal}

University class scheduling problem (UCSP) formalization and definition as a
constraint satisfaction problem (CSP) are presented in section \ref{sec:ProblemFormal}.
In this section would be formally defined:
\begin{itemize}
  \item[$\pm$] problem representation as a \emph{distributed} CSP,
  \item[+] structures of the agents and their contexts,
  \item[$\pm$] candidates assessment,
  \item[-] common goal,
  \item[-] decision process, based on the acceptable candidates,
  \item[-] agents behavior,
  \item[-] agents implementation,
  \item[-] negotiation creation and supervision.
\end{itemize}

%%%%%%%%%%%%%%%%%%%%%%%%%%%%%%%%%%%%%%%%%%%%%%%%%%%%%%%%%%%%%%%%%%%%%%%%%%%%%%%%

% The problem definition in section \ref{sec:ProblemFormal} is already one step
% close to a distributed one; it just lacks explicitly connecting the constraints
% within an agent.

Each participant $p$ (professor, group or classroom) is represented in the negotiation
by a single agent $a_p$, that holds person-specific restrictions and optimization criteria
in its contexts: $ \{\xi^a_i\} \in a_p$. The internal contexts contain to
the restrictions, defined in section \ref{sec:ProblemFormal}:
\begin{align*}
   \text{Capabilities}      &\hbox{ --- } \restrC
&  \text{Time consistency}  &\hbox{ --- } \restrT
\\ \text{Obligations}       &\hbox{ --- } \restrS
&  \text{Preferences}       &\hbox{ --- } \restrW
\end{align*}

The \emph{external} context is not person-specific, but is implemented in the
same way for all the negotiating agents. As mentioned before, it is a mean for the
agents to pursue a common goal (section \ref{sec:CommonGoal}) and ensure
\emph{eq:coh-fun-independ} property.



\subsubsection{Contexts Formalization}
%%%%%%%%%%%%%%%%%%%%%%%%%%%%%%%%%%%%%%%%%%%%%%%%%%%%%%%%%%%%%%%%%%%%%%%%%%%%%%%%

The constraints are defined in the contexts in a more generic way than in problem
formalization. They handle \emph{pieces of information},
that are candidates' underlying classes and \emph{context-specific knowledge},
joined together.
A constraint in context $\ctx$ of agent $a$ can have two forms:
\begin{itemize}[leftmargin=2cm]
  \item[binary] $ \xi_{i_2}^{\ctx^a} : \left< x,y \right> \mapsto \maybe \codom_\ctx $
  \item[whole]  $ \xi_{i_w}^{\ctx^a} : \{i_*\} \mapsto \codom_\ctx $
  \item[] where $x, y$ and $i_*$ are \emph{pieces of information};
    the $\codom$ of the constraint functions depends on the context $\ctx$.
    $\maybe v$ can have two underlying values: Just $v$ and Nothing.
    Nothing is used to denote that the binary relation doesn't handle pieces of information
    of given types.
\end{itemize}

As described in section \ref{sec:coherence}, the contexts require their
constraint functions to have following codomains:
\begin{align*}
   \text{Capabilities}      &\hbox{ --- } \{0, 1\}
&  \text{Time consistency}  &\hbox{ --- } \{-1, 0, 1\}
\\ \text{Obligations}       &\hbox{ --- } \{0, 1\}
&  \text{Preferences}       &\hbox{ --- } [0, 1]
\\ \text{External}          &\hbox{ --- } [0, 1] &
\end{align*}


Each context must define \emph{combination functions} $\eta$, that constructs a
single coherence value out of the values, yielded by the constraints.

Then coherence of candidate $\tilde c$ at context $\ctx^a$
is calculated as \\ $\coh^{\ctx^a}(\tilde c) = \eta(\{b^k\}, \{w^k\})$, where
\begin{align*}
  \{b^k\}  &= \{b_{ij}^k | \forall \text{pieces of information } i, j | i \not= j \} \\
  b_{ij}^k &= \xi_{k_2}^{\ctx^a}(i,j) \\
  &\\
  w^k      &= \xi_{k_w}^{\ctx^a}(\{\text{pieces of information}\})
\end{align*}

\todo\red{: propose function $\eta$.}

\medskip

\noindent
In order for a candidate $\tilde c$ to be accepted at the context $\ctx$
of agent $a$, it's coherence must not be less than context-specific threshold,
that can depend on negotiation duration:
$$ \coh^{\ctx^a}(\tilde c) \geq \rho^{\ctx^a}(\tau) $$

\bigskip

\noindent
A context $\ctx$ of agent $a$ is defined as tuple, containing its
context-specific knowledge, relations, combination function and threshold.

$$ \ctx^a \sim \left< \mathit{knowledge^{\ctx^a}},
                      \{\xi_{i_2}^{\ctx^a}\},
                      \{\xi_{i_w}^{\ctx^a}\},
                      \eta_\ctx,
                      \rho^{\ctx^a}(\tau)
               \right> $$





\subsubsection{Agent Formalization}
%%%%%%%%%%%%%%%%%%%%%%%%%%%%%%%%%%%%%%%%%%%%%%%%%%%%%%%%%%%%%%%%%%%%%%%%%%%%%%%%

\noindent
An agent $a$ is a triple $\left< \behaviour_\act^a,
                                 \behaviour_\react^a,
                                 \state
                          \right>$, where
\begin{align*}
  &\behaviour_\act^a   : \state \mapsto \action \\
  &\behaviour_\react^a : \state \times \msg \mapsto \action \\
  &\mathit{contexts}_a = \left< \mathit{capabilities_a},
                                \mathit{tconsistency_a},
                                \mathit{obligations_a},
                                \mathit{preferences_a},
                                \mathit{external_a}
                           \right>\\
  &\mathit{contexts}_a \in \state
\end{align*}

Behavior functions were introduced in section \ref{sec:NegotiatingAgents};
their implementation is discussed in section \ref{sec:AgentBehavior}.

\todo\red{: define internal and combined coherences (?? common goal ??)}

% Each context defines it's context-specific knowledge, constraints of two kinds
% and a threshold. Constraints codomains depend on the context and are already
% described in section \ref{sec:ProblemFormal}.
% The constraints can have two forms (but both can be represented in the second one):
% \begin{itemize}
%   \item[binary] $\left< x, y \right> \mapsto \mathrm{codomain}$
%   \item[]
% \end{itemize}


% Let a \emph{negotiating coherence-based} agent $a$ be an entity that defines
% the following operations over it:
% \begin{itemize}
%   \item[\textit{contexts}]
% \end{itemize}


% For each participant professor, group, and classroom an agent of the corresponding
% role is defined

% As defined in section \ref{sec:ProblemFormal}, there are


% The restrictions from problem definition in section \ref{sec:ProblemFormal}
% are already defined, taking into account the agent.

\todo







%
