\documentclass[../header]{subfiles}

\begin{document}

\providecommand{\rootdir}{..}

\def\Re{\mathbb{R}}

\def\domain{\mathrm{domain}}

\def\domain{\mathrm{domain}}
\def\pop{\mathrm{pop}}

\def\behaviour{\mathrm{behaviour}}
\def\act{\mathrm{act}}
\def\react{\mathrm{react}}
\def\state{\mathrm{state}}
\def\action{\mathrm{action}}
\def\msg{\mathrm{message}}

\def\coh{\mathrm{coh}}
\def\cohi{\mathrm{\widetilde{coh}}}
\def\rel{\mathrm{rel}}
\def\fold{\mathit{fold}\,}

\def\restrC{\accentset{C}{\xi}}
\def\restrT{\accentset{T}{\xi}}
\def\restrS{\accentset{S}{\xi}^p_i}
\def\restrW{\accentset{W}{\xi}^p_i}

\def\ctx{\mathit{ctx}}
\def\codom{\mathrm{codomain}}
\def\maybe{\mathrm{Maybe\,}}

\def\TRUE{\mathit{TRUE}}

\newcommand{\mkCEUMapGraph}[1][]{
  \begin{tikzpicture}[
    every node/.style={draw, circle, inner sep=0pt,
                       text width=1.5em, align=center},
    node IT/.style={},
    node SN/.style={},
    node CR/.style={},
    node H/.style ={},
    node A/.style ={},
    node SK/.style={},
    node CZ/.style={},
    node G/.style ={},
    node P/.style ={},
    #1
    ]

  \node[node IT] (IT) at (0,0)   {IT};
  \node[node SN] (SN) at (1,0)   {SN};
  \node[node CR] (CR) at (2,-1)  {CR};
  \node[node H]  (H)  at (3,0)   {H};
  \node[node A]  (A)  at (1,1)   {A};
  \node[node SK] (SK) at (3,1)   {SK};
  \node[node CZ] (CZ) at (2,2)   {CZ};
  \node[node G]  (G)  at (1,3)   {G};
  \node[node P]  (P)  at (3,3)   {P};

  \draw (G) -- (P);
  \draw (G) -- (CZ);
  \draw (G) -- (A);
  \draw (P) -- (CZ);
  \draw (P) -- (SK);
  \draw (CZ) -- (A);
  \draw (CZ) -- (SK);
  \draw (A) -- (SK);
  \draw (A) -- (H);
  \draw (A) -- (SN);
  \draw (A) -- (IT);
  \draw (SK) -- (H);
  \draw (IT) -- (SN);
  \draw (SN) -- (H);
  \draw (SN) -- (CR);
  \draw (H) -- (CR);

\end{tikzpicture}
}


\subsubsection{Solving Graph Coloring Problem with Genetic Algorithm}
% Define genes and fitness, give examples of chromosomes.
% And immediately a possible solution, without details.

Graph coloring problem was described in section \ref{sec:CSP-definition}
and graph example can be found on figure \ref{fig:ColoringGraph}.
In order to solve the problem with GA, we would need to encode variables
with \emph{genes} and solutions with \emph{chromosomes}.

All the genes represent same type of variable --- color, with domain of 4 values.
The chromosomes (sequences of genes) represent the colors of graph's nodes.
It's common to use \texttt{Strings} (sequences of characters) as chromosomes.
An example of possible colors assignations is presented on figure
\ref{fig:ColoringGraph-ValsExample}.

\emph{Fitness} function for the problem would be
$\dfrac{1}{1 + \mathrm{broken\,constraints}}$. % with codomain $(0,1]$



% For the graph on figure \ref{fig:ColoringGraph}, one of possible colors
% assignations is:

\begin{figure}
  \centering
  \begin{subfigure}{0.6\textwidth}
    \centering
    \begin{tabular}{l||c|c|c|c|c|c|c|c|c|}
      Node  & IT & SN & CR & H & A & SK & CZ & G & P \\
      Color &  r &  y &  b & r & b &  g &  y & g & r
    \end{tabular}
    \\ (\textbf{r}ed, \textbf{g}reen, \textbf{b}lue, \textbf{y}ellow)
    \\[1em] The corresponding chromosome would be \\ {\large ``rrbrrgyyr''}
  \end{subfigure}
  \qquad
  \begin{subfigure}{0.3\textwidth}
    \mkCEUMapGraph[node IT/.style={red},
                   node SN/.style={yellow!70!black},
                   node CR/.style={blue},
                   node H/.style ={red},
                   node A/.style ={blue},
                   node SK/.style={green!50!black},
                   node CZ/.style={yellow!70!black},
                   node G/.style ={green!50!black},
                   node P/.style ={red},]
  \end{subfigure}
  \caption{Possible graph coloring problem solution}
  \label{fig:ColoringGraph-ValsExample}
\end{figure}

\end{document}
