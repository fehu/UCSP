\documentclass[../ThesisDoc]{subfiles}

\begin{document}

\providecommand{\rootdir}{..}

\def\domain{\mathrm{domain}}

\def\domain{\mathrm{domain}}
\def\pop{\mathrm{pop}}

\def\behaviour{\mathrm{behaviour}}
\def\act{\mathrm{act}}
\def\react{\mathrm{react}}
\def\state{\mathrm{state}}
\def\action{\mathrm{action}}
\def\msg{\mathrm{message}}

\def\coh{\mathrm{coh}}
\def\cohi{\mathrm{\widetilde{coh}}}
\def\rel{\mathrm{rel}}
\def\fold{\mathit{fold}\,}

\def\restrC{\accentset{C}{\xi}}
\def\restrT{\accentset{T}{\xi}}
\def\restrS{\accentset{S}{\xi}^p_i}
\def\restrW{\accentset{W}{\xi}^p_i}

\def\ctx{\mathit{ctx}}
\def\codom{\mathrm{codomain}}
\def\maybe{\mathrm{Maybe\,}}


\section{Contexts}
\label{sec:solution-contexts}


A \emph{context} represents an aspect for optimization/restriction, considered
by an agent. It defines:
\begin{enumerate}
  \item Context-specific information --- set of \emph{information pieces}
        (see section \ref{sec:CoherenceTheory}), that depends on context logic.
        Might be empty if all the information needed is held inside the relations.
  \item Relations over the information graph $\{\rel\}$
        (with the corresponding combination functions $\{f^\cup_2\}$, if applies).

        There are two types of relations: \emph{binary} and over the \emph{whole graph}.
        The former ones are applied to every pair of nodes and the results are combined
        by folding\footnote{
          In functional programming, fold (or reduce) is a family of higher order
          functions that process a data structure in some order and build a return value.
          Typically, a fold deals with two things: a combining
          function, and a data structure, typically a list of elements.
          The fold then proceeds to combine elements of the data structure using
          the function in some systematic way. (\url{https://wiki.haskell.org/Fold})
        } using an \emph{associative binary operation}\footnote{%
          Operator $\bullet$ is associative
          $\Leftrightarrow (a \bullet b) \bullet c = a \bullet (b \bullet c)$
            \label{footnote:binary-operation}
        } $f^\cup_2$ (for each relation).
        The latter, as follows from the name, are defined over whole information graph,
        thus permitting more flexible evaluation.
        \begin{itemize}
          \item \emph{Over whole graph}.
            \begin{align*}
              & \rel: \lbrace I \rbrace \mapsto \Re \times \Details \\
              & \coh[\rel](\tilde{c}) = rel(\tilde{c})
            \end{align*}
          \item \emph{Binary}.
            \begin{align*}
              & \rel: I \times I \mapsto \Re \times \Details \\
              & f^\cup_2: (\Re \times \Details) \times
                          (\Re \times \Details)
                   \mapsto \Re \times \Details \\[0.3em]
              & \coh[\rel](\tilde{c}) = \fold (f^\cup_2[\rel]) ~
                  \left(\lbrace \rel(c_a,c_b) ~|~ \forall\, c_a,c_b \in \tilde{c}
                                          , ~ a \not= b \rbrace\right)
            \end{align*}
          \item[\emph{where}] $I$ --- \emph{information piece}. \\\noindent
                              $\tilde{c} \sim \{c\}$ --- \emph{candidate}
                              (set of \emph{classes}).
                              \underline{Class $c$ is an \emph{information piece}.}

        \end{itemize}
  \item Relations combination associative\textsuperscript{\ref{footnote:binary-operation}}
        function $f^\cup$, that is used to combine coherence values, assessed by
        individual relations. The result is \emph{coherence} at the \emph{context}.
        \begin{align*}
          & f^\cup: (\Re \times \Details) \times
                    (\Re \times \Details)
             \mapsto \Re \times \Details \\
          & \mathrm{assess}[\mathrm{context}]: \Candidate \mapsto
                                               \Re \times \Details
          \\[0.3em]
          & \mathrm{assess}[\mathrm{context}](\tilde{c}) =
            \fold (f^\cup) ~ \left(
            \lbrace \coh[\rel](\tilde{c}) ~|~
                    \forall\, \rel \in \mathrm{relations}(\mathrm{context})
            \rbrace\right)
        \end{align*}

  \item Context-specific \emph{Details}, that describe coherence evaluation process.
  \item Context-specific threshold $\Theta$, that separates \emph{context-coherent} values.
\end{enumerate}

\begin{align*}
  & \coh[\mathrm{context}]: \Candidate \mapsto \Re \times
                            \Bool \times \Details
  \\[0.3em]
  & \coh[\mathrm{context}](\tilde{c}) =
      \letIn{ \left<x,d\right> = \mathrm{assess}[\mathrm{context}](\tilde{c}) }
            {\left<x, x \geq \Theta, d\right>}
\end{align*}


\medskip
\noindent
Contexts may have restrictions, that would compromise a potential solution
because of lack of information during assessment. A good example is
\emph{minimum classes} relation for \emph{professors}, that would force
corresponding agents to immediately reject propositions, that don't have enough
professor's classes, without considering a possibility of joining them to achieve
full constraints compliance.

Coherence \textbf{assessment \emph{modes}} are introduced to counter the problem.
All coherence assessment functions should provide \emph{mode} argument, that
\begin{enumerate}
  \item Is \underline{passed to each context}, when filtering candidates.
  \item Is \underline{passed to each relation}, when assessing a candidate at a context.
  \item Can \underline{change relation evaluation}, weakening or disabling the
        restriction described.
\end{enumerate}

\crule{0.5}

\noindent
The contexts stand for different constraints and can be divided into
\begin{itemize}[leftmargin=2cm]
  \item[Common:] same for all agents, represent \emph{class coherence}:
    \begin{itemize}
      \item \emph{capabilities}: class structure ($\restrC$);
      \item \emph{time consistency}: relations between classes ($\restrT$);
    \end{itemize}
  \item[Internal:] personal configuration, unique for each agent:
    \begin{itemize}
      \item \emph{obligations}: strong restrictions ($\restrW$);
      \item \emph{preferences}: weak restrictions ($\restrS$);
    \end{itemize}
  \item[External:] communicates with the ``neighboring'' agents and fetches
    their opinions about the candidates, thus cooperating towards the
    \emph{common goal}. It doesn't depend on agent's role or
      any agent-specific knowledge. It unites the constraints of various agents.
\end{itemize}


\subsection{Class Coherence}
\subfile{\rootdir/Solution/Contexts/ClassCoherence}
\subsection{Internal Contexts}
\subfile{\rootdir/Solution/Contexts/InternalCoherence}
\subsection{External}
\subfile{\rootdir/Solution/Contexts/ExternalCoherence}


\end{document}
