% % % % % % % % % % % % % % % % % % % % % % % % % % % % % % % % % % % % % % V
\section{Agents}

\begin{frame}{Agents}
  The notion of agent appears in Aristotle's works
   \begin{quote}
     Entity that acts with a purpose, within a social context.
   \end{quote}\\\medskip
  The prætorian roman law defined an agent as
    \begin{quote}
      A person who acts on behalf of a principal for
      a specific purpose and under limited delegation of authority and
      responsibility.
    \end{quote}\\\medskip
  The earliest use of the term agent in AI was
    \begin{quote}
      A program that is capable of executing an action vicariously.
    \end{quote}
\end{frame}

% ?? Solve CSPs with Agents ??

\begin{frame}
  \centering
  \begin{block}{An Agent}
    is \emph{computer system}, that
    \begin{enumerate}
      \item has a degree of autonomy in determining its behavior,
      \item interacts with humans and or other agents,
      \item perceives the environment and reacts to it, and
      \item exhibits a goal directed behavior.
    \end{enumerate}
  \end{block}
  \begin{block}{A Negotiation}
     is a process of \underline{communication} between heterogeneous agents
     with goal of resolving some common problem.
  \end{block}
  \begin{block}{A Negotiating Agent}
     is an \emph{isolated} \emph{proactive} computational entity,
     capable of sending and receiving messages.
  \end{block}
\end{frame}

\begin{frame}{Agents}
  \begin{block}{Behaviour}
    An agent is defined by it's behaviour:
    \begin{align*}
      \behaviour_\act   &: \state \mapsto \action \\
      \behaviour_\react &: \state \times \msg \mapsto \action
    \end{align*}
    % The agents must use a common \emph{communication protocol}, to ensure
    % understanding between agents of the same or different roles.
  \end{block}
  \begin{block}{Role}
    A role describes whom or what an agents represents in the negotiation and
    defines \emph{behaviour archetype}.
  \end{block}
  \begin{block}{Implementation}
    Agents are implemented in \underline{Haskell} using
    \emph{Software Transactional Memory} (STM)
    --- a promising concurrency paradigm.
  \end{block}
\end{frame}
