\documentclass[ThesisDoc]{subfiles}

\providecommand{\rootdir}{.}

\def\domain{\mathrm{domain}}

\def\domain{\mathrm{domain}}
\def\pop{\mathrm{pop}}

\def\behaviour{\mathrm{behaviour}}
\def\act{\mathrm{act}}
\def\react{\mathrm{react}}
\def\state{\mathrm{state}}
\def\action{\mathrm{action}}
\def\msg{\mathrm{message}}

\def\coh{\mathrm{coh}}
\def\cohi{\mathrm{\widetilde{coh}}}
\def\rel{\mathrm{rel}}
\def\fold{\mathit{fold}\,}

\def\restrC{\accentset{C}{\xi}}
\def\restrT{\accentset{T}{\xi}}
\def\restrS{\accentset{S}{\xi}^p_i}
\def\restrW{\accentset{W}{\xi}^p_i}

\def\ctx{\mathit{ctx}}
\def\codom{\mathrm{codomain}}
\def\maybe{\mathrm{Maybe\,}}


\begin{document}

\section{ Assumptions and Improvements }
\label{sec:improvements}

In the solution, presented by this \thisdoc\ some assumptions and
simplifications were made. In this section they are discussed and some
improvement/extension propositions are made.

% % % % % % % % % % % % % % % % % % % % % % % % % % % % % % % % % % % % % % % %
\subsection{Timetable}
In current implementation time slots have no spaces between them and it is
assumed that each class includes the time needed by the persons to
move from room to room.

\begin{figure}[h]
  \centering
  \begin{tabular}{|c||c|c|c|c|c|c|}
    \hline & Mon & Tue & Wed & Thu & Fri & Sat \\
    \hhline{|=#=|=|=|=|=|=|}
    08:00 -- 08:30 & x & & & w & & \\\hline
    08:30 -- 09:00 & x & & & w & & \\\hline
    09:00 -- 09:30 & x & & & z & & \\\hline
    09:30 -- 10:00 & y & & & z & & \\\hline
    10:00 -- 10:30 & y & & & z & & \\\hline
    $\vdots$\quad~--~\quad$\vdots$ & & & & & & \\\hline
    21:00 -- 21:30 &   & & & & & \\\hline
    21:30 -- 22:00 &   & & & & & \\\hline
  \end{tabular}
  \caption{Current implementation of \emph{timetable} has
           no spaces between classes. }
  \label{fig:timetable-current}
\end{figure}


\medskip\noindent
Time margins around classes can be added in two ways:
\begin{itemize}
  \item Change timetable slots to include intervals
        (see figure \ref{fig:timetable-intervals-predef}).

  \item Use smaller discretization interval and modify time intersection function:
        either the beginning or end time of one class is in between of
        the \emph{modified} beginning and end time of another,
        where \emph{modified} time has the margin added
        (see figure \ref{fig:timetable-intervals-any}).
\end{itemize}

\begin{figure}[h]
  \begin{subfigure}[b]{.5\textwidth}
    \centering
    \begin{tabular}{|c||c|c|c|}
      \hline & Mon & Tue & $\cdots$ \\
      \hhline{|=#=|=|=|}
      08:30 -- 09:15 & x & & \\\hline
      09:25 -- 10:10 & x & & \\\hline
      10:30 -- 11:15 & y & & \\\hline
      11:25 -- 12:10 & y & & \\\hline
      Lunch          &   & & \\\hline
      13:25 -- 14:10 & z & & \\\hline
      14:20 -- 15:05 & z & & \\\hline
      15:25 -- 16:10 &   & & \\\hline
      $\vdots$\quad~--~\quad$\vdots$ & & & \\\hline
    \end{tabular}
    \caption{Timetable with spaces between time slots and lunch time.}
    \label{fig:timetable-intervals-predef}
  \end{subfigure}
  ~
  \begin{subfigure}[b]{.5\textwidth}
    \centering
    \begin{tabular}{|c||c|c|c|}
      \hline & Mon & Tue & $\cdots$ \\
      \hhline{|=#=|=|=|}
      08:30 -- 08:40 & x & & \\\hline
      08:40 -- 08:50 & x & & \\\hline
      08:50 -- 09:00 & x & & \\\hline
      $\vdots$\quad~--~\quad$\vdots$ & & & \\\hline
      09:50 -- 10:00 & x & & \\\hline
      10:00 -- 10:10 &   & & \\\hline
      10:10 -- 10:20 & y & & \\\hline
      10:20 -- 10:30 & y & & \\\hline
      $\vdots$\quad~--~\quad$\vdots$ & & & \\\hline
    \end{tabular}
    \caption{Using timetable with smaller time slots allows leaving spaces
             between classes.}
    \label{fig:timetable-intervals-any}
  \end{subfigure}
  \caption{}
\end{figure}

% % % % % % % % % % % % % % % % % % % % % % % % % % % % % % % % % % % % % % % %
\subsection{Classes Splitting}

In current implementation a \underline{single class} is created for each
group's discipline, without considering the resulting length
(it only has to fit timetable).

The improvement can be made by modifying ``Day -- Time -- Room''
assignment strategy (see section \ref{sec:solution-DTR}).
In general such strategy would need:
\begin{enumerate*}[1)]
  \item class length limit,
  \item days ``distance'',
\end{enumerate*}
both possibly dependent on the discipline.
The classes for same discipline then should be assigned to the cores all together.

% % % % % % % % % % % % % % % % % % % % % % % % % % % % % % % % % % % % % % % %
\subsection{Full-/Part- Time Professors}
Apart from \emph{full-time} working professors, many universities contract
\emph{part-time} professors, that work only some pre-established hours.
Such professors can have different rules for classes assignment, depending on
the institution.

The distinction between professor types wasn't implemented, because it isn't
general case. However, assuming that \emph{part-time} professors are contracted
to teach classes, that no or few full-time professor can, the classes of former
professor type would usually receive higher \emph{discipline priority} and
therefore assigned first. That strategy corresponds with the fact that part-time
professors have stronger restrictions (in \emph{obligations} internal context)
over classes time, because of limited working hours.

If other behavior is needed when dealing with \emph{part-time} professors, then
it can be achieved by extending \texttt{Professor} role constructor and matching
concrete role in agents' behavior functions.

% % % % % % % % % % % % % % % % % % % % % % % % % % % % % % % % % % % % % % % %
\subsection{Students Representation}

The biggest assumption made in this \thisdoc\ is that the groups are preformed
and \emph{each student belongs strictly to one group}. This permits to handle
groups completely ignoring the underlying students, thus simplifying schedule
negotiation.
The assumption makes it impossible to resolve a large subclass of UCSPs, where
the groups are not fixed, but formed for each discipline from the students
that need it.

Basing the negotiation on \emph{students} would require a new preceding stage:
\underline{group formation}. Groups should be formed dynamically by students in
order to represent them in negotiation with professors. Different strategies
could be used for groups creation, that would depend on group size limits.
In general, initially a minimal set of groups (with maximum students possible)
should be created for each discipline, possibly by students negotiation.
More groups can be created later to separate some students, causing conflicts;
of course all group size restrictions must be respected by newly created
group and its ``donors''. A less radical way to change groups composition is
student transfer.

A group is formed by students to study specific discipline (or a set of
explicitly linked ones). All the classes of a group would have the same professor.
In this negotiation scheme, a group would have no more constraints of its own.
All coherence assessment is passed down to the underlying students and then
combined. On figures \ref{fig:ScheduleHypercube-S} and
\ref{fig:ConnectionMatrixDiff} the differences in agents connection organization
are presented.

As in the original negotiation, a group should be responsible for candidate
creation and placement, representing the underlying students in \underline{all}
the communications. Therefore the negotiation scheme would remain unchanged
(see section \ref{sec:solution} and figure \ref{fig:solution-flow}).
The modification would only affect group proactive behavior:
\begin{enumerate}
  \item Internal coherence assessment would interrogate student agents.
  \item A group could not only change the candidate, but also request changes
        to groups compositions.
\end{enumerate}

\medskip\noindent
Proxying would complicate the resulting relations between the agents, so
some tests are needed to make further implementation suggestions.


% % % % % % % % % % % % % % % % % % % % % % % % % % % % % % % % % % % % % % % %
\begin{figure}[h]
  \begin{subfigure}{0.36\textwidth}
    \resizebox{\linewidth}{!}{
      \subfile{\rootdir/img/ConnectionMatrix2/WithComments.tikz}
    }
    \caption{Originally, the connections between groups and classes are
             defined by classes (class-cores).}
  \end{subfigure}
  ~
  \begin{subfigure}{0.64\textwidth}
    \resizebox{\linewidth}{!}{
      \subfile{\rootdir/img/ConnectionMatrix2/WithStudents.tikz}
    }
    \caption{In the new model, each group connects with strictly one professor
             and a set of underlying students.}
    \label{}
  \end{subfigure}
  \caption{}
  \label{fig:ConnectionMatrixDiff}
\end{figure}


\begin{figure}[b]
  \centering
  \resizebox{\textwidth}{!}{
    \subfile{\rootdir/img/ScheduleHypercube/SGRPT.tikz}
  }
  \caption{Schedule space with \emph{students}. Group space consists of
           sets of students that perform coherence evaluation. }
  \label{fig:ScheduleHypercube-S}
\end{figure}


% % % % % % % % % % % % % % % % % % % % % % % % % % % % % % % % % % % % % % % %
\subsection{Classrooms Representation}
It is possible to make the classrooms a represented entity, by creating
corresponding agent \emph{role} and changing the \emph{class-cores}.
Then the classrooms can have internal contexts of their own, and their
\emph{opinions} would be included by external coherence evaluation.
Agents connections in case of students and classrooms representation are shown
on figure \ref{fig:ConnectionMatrixAll}.
\begin{figure}[h]
  \resizebox{\textwidth}{!}{
    \subfile{\rootdir/img/ConnectionMatrix2/WithStudentsAndRooms.tikz}
  }
  \caption{Agents connections in case of full representation.}
  \label{fig:ConnectionMatrixAll}
\end{figure}

\end{document}
