\documentclass[../header]{subfiles}

\begin{document}

\providecommand{\rootdir}{..}

\def\Re{\mathbb{R}}

\def\domain{\mathrm{domain}}

\def\domain{\mathrm{domain}}
\def\pop{\mathrm{pop}}

\def\behaviour{\mathrm{behaviour}}
\def\act{\mathrm{act}}
\def\react{\mathrm{react}}
\def\state{\mathrm{state}}
\def\action{\mathrm{action}}
\def\msg{\mathrm{message}}

\def\coh{\mathrm{coh}}
\def\cohi{\mathrm{\widetilde{coh}}}
\def\rel{\mathrm{rel}}
\def\fold{\mathit{fold}\,}

\def\restrC{\accentset{C}{\xi}}
\def\restrT{\accentset{T}{\xi}}
\def\restrS{\accentset{S}{\xi}^p_i}
\def\restrW{\accentset{W}{\xi}^p_i}

\def\ctx{\mathit{ctx}}
\def\codom{\mathrm{codomain}}
\def\maybe{\mathrm{Maybe\,}}

\def\TRUE{\mathit{TRUE}}



\subsection{Selective Hyper-Heuristic}
The main idea of the method is dynamic selection of the most suitable solution method
from a set of algorithms or heuristics \cite{CSPhypHeur}. It is achieved
using Learning Classifier Systems (LCS), that are
\begin{quote}
  $\dots$ adaptive rule-based systems that automatically build the set of rules
  they manipulate. These rules are called classifiers, and they
  allow the system to respond to different situations of the
  environment \cite{CSPhypHeur}.
\end{quote}

The authors use rather simple \emph{Arc Consistency Algorithm \#3} (AC3), noting
that other constraint propagation method can be used instead.
The classifiers serve to \underline{dynamically select} one of four
\emph{variable ordering heuristics} for the algorithm.
\emph{Hyper-heuristic} is the set of classifiers in the system \cite{CSPhypHeur}.

In short, AC3 operates \emph{variables} and their \emph{domain}. The former are
represented by graph \underline{nodes}. Graph edges represent constraint. The
algorithm checks variables (nodes of the graph) one-by-one, trying to assign a value,
consistent with the existing nodes assignation (pruning its domain).
A similar technique, applied to agents, is described in section \label{sec:CSP-Agents}.

The hyper-heuristic is used to select \underline{next variable}, that AC3 should
process. Value assignment is done according to the Min-Conflicts heuristic \cite{CSPhypHeur}.



\end{document}
