\documentclass[ThesisDoc]{subfiles}

\providecommand{\rootdir}{.}

\def\domain{\mathrm{domain}}

\def\domain{\mathrm{domain}}
\def\pop{\mathrm{pop}}

\def\behaviour{\mathrm{behaviour}}
\def\act{\mathrm{act}}
\def\react{\mathrm{react}}
\def\state{\mathrm{state}}
\def\action{\mathrm{action}}
\def\msg{\mathrm{message}}

\def\coh{\mathrm{coh}}
\def\cohi{\mathrm{\widetilde{coh}}}
\def\rel{\mathrm{rel}}
\def\fold{\mathit{fold}\,}

\def\restrC{\accentset{C}{\xi}}
\def\restrT{\accentset{T}{\xi}}
\def\restrS{\accentset{S}{\xi}^p_i}
\def\restrW{\accentset{W}{\xi}^p_i}

\def\ctx{\mathit{ctx}}
\def\codom{\mathrm{codomain}}
\def\maybe{\mathrm{Maybe\,}}


\begin{document}

\chapter{Constraint Satisfaction Problems}
\label{chapter:csp}

There are many problems that require positioning or assigning something,
respecting established \emph{restrictions}. \emph{Graph coloring} and
\emph{n-queen} chess problem are classical constraint satisfaction problems (CSPs).


The graph coloring problem comes from cartography, where it was needed to color
countries on political maps, in such a way that no country had a land border with
a country of the same color. It was found that \textbf{any} map can be colored
with only four colors.

The n-queen problem is known in chess as \emph{eight queen puzzle}.
Queens in chess can move/attack to/at any square,
that is in a the same row, column or diagonal with the queen.
In the puzzle one needs to place eight (the size of a chess desk)
queens on the desk, so that none of them is threatened by another.
N-queens problems is a generalization of that puzzle, where $n$ queens need
to be placed on a $n \times n$ desk.

Constraint satisfaction problems are found in many areas:
machine vision, natural language processing, theorem proving,
planning and in our problem --- scheduling \cite{MAS}.

% % % % % % % % % % % % % % % % % % % % % % % % % % % % % % % % % % % % % % % %
\subfile{CSP/Formal}
% % % % % % % % % % % % % % % % % % % % % % % % % % % % % % % % % % % % % % % %

\section{Solution Methods}
  Solving these problems usually presents difficulties due to the amount of
possible combinations to consider for a solution. Let's have a look at
some university schedule.
  For example, six working days in a week and twelve time slots every day
(every hour from 8:00 to 20:00), would yield $6 \times 12 = 72$ options
to place each class.
  Given that each group needs, for example, five different disciplines to be assigned,
there would be $\multibinom{72}{5} = \num[group-separator={,}]{18474840}$
possible classes assignments for each group,
even without considering professors and classrooms.

\medskip
\noindent
The researchers have been looking for means of solving such problems within
reasonable time. During the last years different algorithms and techniques where
developed, such as
\emph{dynamic constraint satisfaction based on extension particle swarm
      optimization algorithm} \cite{CSPswarm},
\emph{dynamic state bounding} \cite{CSPdynStateBound},
\emph{conflict-vector detection} \cite{CSPtimetable},
\emph{neural networks} \cite{CSPneuro},
\emph{ant colony optimization} \cite{CSPcunningACO, CSPlimmemACO},
\emph{selective hyper-heuristics} \cite{CSPhypHeur},
\emph{quantum-inspired genetic algorithm} \cite{QuantumGeneticAlgorithm}
and \emph{agents} \cite{CSPagent2013, CSPagent2014, DCSPagent1998}.

\medskip

Those methods do not seek for the optimal solutions, focusing instead on
``rather good'' ones, that can be obtained within reasonable time using admissible
computing resources.

\bigskip

\noindent
Some agent solution methods are presented in section \ref{sec:CSP-Agents}.

% % % % % % % % % % % % % % % % % % % % % % % % % % % % % % % % % % % % % % % %
\subfile{CSP/GA}
\subfile{CSP/QGA}
% % % % % % % % % % % % % % % % % % % % % % % % % % % % % % % % % % % % % % % %

\section{Distributed problems}
Agents approach allows solving CSPs in a distributed manner by distributing
the problem knowledge and constraints
\cite{DCSPagent1998, DCSP2013, CSPagent2014, MAS, MAS-Survey}.

\begin{displayquote}[\cite{MAS-Survey}] % \cite[p.~2]{MAS-Survey}
  For example, a constraint satisfaction problem can often be
  decomposed into several not entirely independent
  subproblems that can be solved on different processors. $\dots$
  MAS (Multiagent System) allows the sub-problems of a constraint satisfaction
  problem to be subcontracted to different problem solving agents with their own
  interests and goals.
\end{displayquote}

\noindent
Such problems are \emph{Distributed Constraints Satisfaction Problems} (DCSPs).
\begin{displayquote}[\cite{MAS}] % \cite[p.~4]{MAS}
  In a distributed CSP, each variable is owned by a different agent. The goal is
  still to find a global variable assignment that meets the constraints, but each agent
  decides on the value of his own variable with relative autonomy. While he does
  not have a global view, each agent can communicate with his neighbors in the
  constraint graph.
\end{displayquote}


University class scheduling problem fits into a the DCSP class:
\begin{itemize}
  \item Some constraints are agent-specific: strong restrictions and
        personal preferences.
  \item Classes assignments in form of \emph{candidates} are distributed between
        the negotiating agents (representing groups and professors).
\end{itemize}

The original UCSP is decomposed in a set of sub-problems, represented by some
participants. A sub-problem consist in resolving constraints over an agent's
\emph{candidate}, while considering the \emph{opinions} of \emph{neighboring}
agents.

The sub-problems are then recomposed into a schedule, resolving any conflicts
between the sub-solutions with the established rules.

\end{document}
