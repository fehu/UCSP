\documentclass[../../header]{subfiles}

\begin{document}

\providecommand{\rootdir}{../..}

\def\domain{\mathrm{domain}}

\def\domain{\mathrm{domain}}
\def\pop{\mathrm{pop}}

\def\behaviour{\mathrm{behaviour}}
\def\act{\mathrm{act}}
\def\react{\mathrm{react}}
\def\state{\mathrm{state}}
\def\action{\mathrm{action}}
\def\msg{\mathrm{message}}

\def\coh{\mathrm{coh}}
\def\cohi{\mathrm{\widetilde{coh}}}
\def\rel{\mathrm{rel}}
\def\fold{\mathit{fold}\,}

\def\restrC{\accentset{C}{\xi}}
\def\restrT{\accentset{T}{\xi}}
\def\restrS{\accentset{S}{\xi}^p_i}
\def\restrW{\accentset{W}{\xi}^p_i}

\def\ctx{\mathit{ctx}}
\def\codom{\mathrm{codomain}}
\def\maybe{\mathrm{Maybe\,}}


%%%%%%%%%%%%%%%%%%%%%%%%%%%%%%%%%%%%%%%%%%%%%%%%%%%%%%%%%%%%%%%%%%%%%%%%%%%%%%%%
\section{Group Agents}

\emph{Group agents} play active role in the negotiation.
Initially, each group generates a \emph{random candidate}.
Generation process ensures that \emph{class coherence} constraints
($\restrC$ and $\restrT$) are satisfied, thus implementing \textit{Capabilities}
and \textit{Time Consistency} contexts at once. It consists of three stages:
\begin{enumerate}
  \item \emph{Class-Cores Pool} generation.
  \item Generation of a \emph{class-cores} from the \emph{pool}.
  \item \emph{Day-Time-Room} assignment.
  %  Values are generated randomly for the
  %       entire set of class-cores, respecting time restrictions $\restrT$.
  %       % Classes duration is specified alongside each discipline.
\end{enumerate}


\noindent
After estimating generated candidate's coherence, the agent either tries to put
it into the \emph{common schedule variable} or generates a new one.


%%%%%%%%%%%%%%%%%%%%%%%%%%%%%%%%%%%%%%%%%%%%%%%%%%%%%%%%%%%%%%%%%%%%%%%%%%%%%%%%

\begin{figure}[H]
  \centering
  \subfile{\rootdir/img/ConnectionMatrix/Candidates.tikz}
  \caption[Class cores and candidates]
          {The \emph{class cores} define classes' connections. Candidates are
           created by groups and refer the professors, mentioned in the
           underlying classes. }
  \label{fig:class-cores}
\end{figure}

%%%%%%%%%%%%%%%%%%%%%%%%%%%%%%%%%%%%%%%%%%%%%%%%%%%%%%%%%%%%%%%%%%%%%%%%%%%%%%%%
\subsection{Class-Cores}
\label{sec:solution-CC}

A class core is considered to be the immutable part of a class, that doesn't
change during its existence. It links \emph{group} and \emph{professor} agents
through a \emph{discipline}, respecting the \emph{capability} restrictions $\restrC$
(see figure \ref{fig:class-cores}).
It can also be seen as an ``abstract'' class, with no time or classroom assigned.

A \emph{Class-Cores Pool} is a lazy random sequence of \emph{class-cores},
that contains a class for each discipline, needed by the group.

\emph{Pool} creation requires knowledge of the existing professor agents with
the disciplines, that the represented person can teach.
\begin{enumerate}
  \item For each discipline needed select professors, that can teach it.
        Randomize professors lists.
  \item Lazily generate all possible combinations \emph{professor} -- \emph{discipline}.
  \item When getting next combination, assign the generating \emph{group}.
\end{enumerate}

%%%%%%%%%%%%%%%%%%%%%%%%%%%%%%%%%%%%%%%%%%%%%%%%%%%%%%%%%%%%%%%%%%%%%%%%%%%%%%%%
\subsection{Day -- Time -- Room}
\label{sec:solution-DTR}

In order to obtain concrete classes, the cores must be completed with day, room,
beginning time and duration (end time), while respecting both discipline duration
requirements and time consistency $\restrT$.

Assigning the missing values can be seen as mapping (randomly) the set of cores onto
3-dimensional space: $\left< \text{day, time, room} \right>$.
In order to do it, a random generator is used. It uses a history variable to
store generated values and test time consistency of the new ones against the former.
The following process is repeated for each class core (using same generator).
\begin{enumerate}
  \item Generate random \emph{day}.
  \item Select random \emph{classroom}.
  \item Generate \emph{beginning time}.
  \item Get class duration from the \emph{discipline}, contained in argument class-core.
        Calculate \emph{end time}.
  \item Test \emph{end time} consistency (upper bound).
  \item Test \emph{time consistency} of the values from (1-4) using the history.
  \item If new values are consistent, add them to history, assign to class-core
        and return it.
        Otherwise, repeat from (1).

\end{enumerate}
After the assignment has been done for all the class-cores in the given set,
the history variable is reset.

\begin{note}
  With rather big set of classrooms and adequate time discretization, the generated
  values can be considered unique with a rather high probability.
\end{note}


%%%%%%%%%%%%%%%%%%%%%%%%%%%%%%%%%%%%%%%%%%%%%%%%%%%%%%%%%%%%%%%%%%%%%%%%%%%%%%%%

\subsection{Candidate placement}
\begin{itemize}
  \item If generated candidate was found coherent by all contexts,
        then the agent must try to put it in the \emph{common schedule}
        (see section \ref{sec:solution-schedule-var} and figure \ref{fig:candidate-conflict}).
        \begin{itemize}
          \item In case of successful candidate placement, the agent ``goes to sleep'',
                until disturbed by \emph{yield} or \emph{continuation} demands.
          \item In case of conflicts with any candidates, already existing
                in current schedule, the conflict is resolved as described
                in section \ref{sec:solution-compare}.
                \begin{itemize}
                  \item It the conflict is resolved in favor of the agent,
                        then it tries again to put the candidate into the
                        schedule, providing evidence of its superiority.
                \end{itemize}
        \end{itemize}
  \item If the candidate was found incoherent or has lost a conflict competition,
        then the candidate must be changed, as described in section
        \ref{sec:solution-change}.
\end{itemize}

An illustration of candidate placement intention is presented on figure
\ref{fig:candidate-conflict}.
Let each row represent some group's candidate as its class-cores.
Let's suppose that all the candidates, except for the \red{$g_i$}'s
are already in schedule, and agent \red{$g_i$} \emph{tries to put}
its candidate $\tilde{c}_i$ into the schedule.
The candidate $\tilde{c}_i$ might have \emph{conflicts} over the
placement with other candidates, in the example those would be
some candidates of \blue{\{$g_*\}$}, including \blue{$g_2$} and
\blue{$g_n$}. $\tilde{c}_i$ can be placed only if it doesn't conflict
with any other candidate or if its creator can \emph{prove} that
$\tilde{c}_i$ is ``better'' than any conflicting candidate.

%%%%%%%%%%%%%%%%%%%%%%%%%%%%%%%%%%%%%%%%%%%%%%%%%%%%%%%%%%%%%%%%%%%%%%%%%%%%%%%%

\begin{figure}[H]
  \centering
  \subfile{\rootdir/img/ConnectionMatrix/Conflict.tikz}
  \caption{ Example of candidate placement intention }
  \label{fig:candidate-conflict}
\end{figure}

%%%%%%%%%%%%%%%%%%%%%%%%%%%%%%%%%%%%%%%%%%%%%%%%%%%%%%%%%%%%%%%%%%%%%%%%%%%%%%%%

\subsection{Placement Conflicts Resolution --- Candidates Comparison}
\label{sec:solution-compare}

A conflict over placing a candidate $\tilde{c}_i$ into the schedule arises
between candidates' creators: group agents $g_i$ and $\{g_k\}$.
The latter are group agents, that have proposed the conflicting classes
as part of their candidates.

As it was mentioned earlier, candidates ``quality'' comparison is not only based
on coherence, but also on candidates ``rareness'', based on \emph{discipline priority}.
There is a third measure: \emph{deep} (or \emph{cascade}) coherence.

\medskip
\noindent
These measures have priorities:
$$ \text{Ext. coherence} < \text{Deep ext. coherence} < \text{Discipline priority} $$

\medskip
\noindent
Deep external coherence, assessed in \emph{final} candidate assessment mode
have priority over those, assessed in \emph{preliminary} mode.

\bigskip
\noindent
Agent's goal in this sub-negotiation is providing stronger support for its
candidate. The ``newcomer'', an agent that wants to put its candidate, is
in disadvantage here: it must beat all the conflicting candidates at once in
order to receive a place. The conflicting candidates, already put in the schedule,
keep their places if at least one of them was found better then the ``newcomer''.
It the latter could prove its candidate superiority in every conflict, then
it can demand candidate's placement (attaching victory proofs).
If no new conflict has arisen, the \emph{schedule controller}
(see section \ref{sec:solution-schedule-var}) puts the new candidate into the
common variable, removing the conflicting candidates (given proof) and notifying
corresponding creators with \emph{yield demand}.


% % % % % % % % % % % % % % % % % % % % % % % % % % % % % % % % % % % % % % % %

\subsubsection{External coherence}
Is used for preliminary/default assessment of the candidates.
Evaluation is done with ``preliminary'' context assessment \emph{mode}, that
disables certain relations; it is intended primarily to be used by professor agents.
It is \emph{deep external coherence} of depth 0 --- only own external coherence
(composed of internal coherence assessments by all the mentioned agents) is
considered.


% % % % % % % % % % % % % % % % % % % % % % % % % % % % % % % % % % % % % % % %

\subsubsection{Deep external coherence}
External coherence of depth $N$. Can be assessed in either \emph{preliminary}
or \emph{final} context assessment mode. Deep coherence has higher priority if
assessed in the latter mode.
No class (or its references), belonging to a conflicting candidate, can be used in
the evaluation.
$N$ is the maximum depth of evaluation recursive calls reached.
The process covers all valid routes, avoiding cycles and repetitions.

Each depth level aggregate into consideration more agents, that are at the
``frontier'' of the previous depth. When some group agent $g_i$ assesses the
deep coherence of its candidate $\tilde{c}_i$, it accumulates internal
coherence values of other agents:
\begin{itemize}[leftmargin=2.5cm]
  \item[Depth 0:] internal coherence of the professors,
            mentioned by candidate $\tilde{c}_i$ are accumulated.
  \item[Depth 1:] internal coherence of the group agents,
            that have created candidates, mentioning groups from depth 0.
  \item[$\vdots$]
  \item[Even depth $j$:] internal coherence of the professors,
            mentioned by candidates of the professors from depth $j-1$.
  \item[Depth $j+1$:] internal coherence of the groups,
            that created candidates, mentioning of the professors from depth $j$.
  \item[$\vdots$]
  \item[Depth $N+1$:] Stop criterion.
            All ``new'' coherence values were already retrieved at some previous depth.
\end{itemize}

% % % % % % % % % % % % % % % % % % % % % % % % % % % % % % % % % % % % % % % %

\begin{figure}[H]
  \centering
  \resizebox{0.7\textwidth}{!}{
    \subfile{\rootdir/img/ConnectionMatrix/ConflictDeepCoherence.tikz}
    }
  \caption[Example of deep coherence evaluation]
          {Example of deep coherence evaluation by agent $g_i$.}
  \label{fig:ConflictDeepCoherence}
\end{figure}
\newpage

% % % % % % % % % % % % % % % % % % % % % % % % % % % % % % % % % % % % % % % %

\noindent
\underline{External coherence} value is the result of applying the selected
\emph{internal} coherence values to \emph{common goal} function $\Gamma$.

\medskip

\noindent
An example of deep coherence evaluation by some agent $g_i$ is shown on figure
\ref{fig:ConflictDeepCoherence}. $g_i$ wants to put a new candidate, that is
in conflict with the candidate marked \red{red}.


% % % % % % % % % % % % % % % % % % % % % % % % % % % % % % % % % % % % % % % %

\subsubsection{Candidate ``rareness''}
Sum of ``rare'' disciplines' priorities. Discipline's ``rarity'' (or
\emph{priority}) is ratio
\begin{itemize}
  \item[\textit{of}] groups enrolled to the discipline ($N_G$)
  \item[\textit{to}] professors able to teach it ($N_P$).
\end{itemize}

$$\rho^d = \dfrac{N_G}{N_P}$$

\medskip
\noindent
Discipline $d$ is considered ``rare'' if its priority $\rho_d$ is higher than
some threshold. \\
\noindent
Candidate's ``rareness'' $\rho_{\tilde{c}}$ therefore is
\begin{align*}
  \rho_{\tilde{c}} &= \sum\limits_{d \in D'_{\tilde{c}}}
        \rho^d \mathsmaller{ \sum\limits_{c \in \tilde{c}'_d}
                              \mathit{duration}\, c }\\
  D'_{\tilde{c}} &= \lbrace d ~|~ d ~\mathit{referenced\,by}~ \tilde{c};~
                                \rho_d > \rho_* \rbrace\\
  \tilde{c}'_d &= \lbrace c ~|~ c \in \tilde{c};~ c ~\mathit{is\,class\,for}~ d \rbrace
\end{align*}

%%%%%%%%%%%%%%%%%%%%%%%%%%%%%%%%%%%%%%%%%%%%%%%%%%%%%%%%%%%%%%%%%%%%%%%%%%%%%%%%

\subsection{Change of a candidate}
\label{sec:solution-change}

Triggered either when a candidate was found \emph{not sufficiently coherent}
(see section \ref{sec:solution-better}) immediately after generation
or as a result of loosing candidates conflict.
When the change was caused by a conflict, then agent must consider adversaries'
quality values and try to better them.

Agent should first try to achieve expected quality by changing
\emph{day}, \emph{time} and \emph{classroom} parameters of candidate's classes
(see section \ref{sec:solution-DTR}).
If it is unable to find a satisfying candidate by these changes within some
reasonable negotiation time, then it should generate a candidate with new
\emph{class core} (see section \ref{sec:solution-CC}).

When class cores end up, an error would be raised by the agent, causing the whole
negotiation to terminate.

%%%%%%%%%%%%%%%%%%%%%%%%%%%%%%%%%%%%%%%%%%%%%%%%%%%%%%%%%%%%%%%%%%%%%%%%%%%%%%%%


\end{document}
