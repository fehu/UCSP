\documentclass[../../ThesisDoc]{subfiles}

\begin{document}

\providecommand{\rootdir}{../..}

\def\domain{\mathrm{domain}}

\def\domain{\mathrm{domain}}
\def\pop{\mathrm{pop}}

\def\behaviour{\mathrm{behaviour}}
\def\act{\mathrm{act}}
\def\react{\mathrm{react}}
\def\state{\mathrm{state}}
\def\action{\mathrm{action}}
\def\msg{\mathrm{message}}

\def\coh{\mathrm{coh}}
\def\cohi{\mathrm{\widetilde{coh}}}
\def\rel{\mathrm{rel}}
\def\fold{\mathit{fold}\,}

\def\restrC{\accentset{C}{\xi}}
\def\restrT{\accentset{T}{\xi}}
\def\restrS{\accentset{S}{\xi}^p_i}
\def\restrW{\accentset{W}{\xi}^p_i}

\def\ctx{\mathit{ctx}}
\def\codom{\mathrm{codomain}}
\def\maybe{\mathrm{Maybe\,}}


%%%%%%%%%%%%%%%%%%%%%%%%%%%%%%%%%%%%%%%%%%%%%%%%%%%%%%%%%%%%%%%%%%%%%%%%%%%%%%%%
\section{Professor Agents}
Professor agents are purely reactive --- act only in response to messages.
They don't have own candidates, but are referenced by those of the \emph{groups}.
Candidates report their \emph{opinions} as part of \emph{external coherence}
mechanism. They support solution improvement (see section \ref{sec:solution-better})
and apply it to the assessments.

Because the solution is formed in steps and professor agents play no active role
in it, professors' internal contexts might cause \emph{over-restrictions}, for
example minimal number of classes. In order to consider these cases,
\emph{context assessment modes} were introduced.

Tho modes in use are: \emph{preliminary}, that disables ``unsafe'' contexts' relations,
and \emph{final}.


\end{document}
