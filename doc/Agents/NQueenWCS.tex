\documentclass[../ThesisDoc]{subfiles}


\newcommand{\queensImg}[2][]{%
  \resizebox{\linewidth}{!}{\QueensPicture[#1]{#2}}%
}

\newcommand{\queensSubfig}[2]{
  \begin{subfigure}[t]{#1\hsize}
    \caption{}
    \queensImg{#2}
  \end{subfigure}
}


\newcommand{\showSteps}[4][]{
  \gdef\acc{}
  \gdef\first{1}
  \def\w{#2}
  \def\showBoard{\agentsBoard[#1]}
  \foreach \i in {#3}
    {
      \ifthenelse{\first=1}
                 {\numgdef\first{0}}
                 {\xappto\acc{,}}
      \xappto\acc{\i}
      \queensSubfig{\w}{\expandafter\showBoard\expandafter{\acc}#4}
    }
}



\begin{document}

\providecommand{\rootdir}{..}

\def\domain{\mathrm{domain}}

\def\domain{\mathrm{domain}}
\def\pop{\mathrm{pop}}

\def\behaviour{\mathrm{behaviour}}
\def\act{\mathrm{act}}
\def\react{\mathrm{react}}
\def\state{\mathrm{state}}
\def\action{\mathrm{action}}
\def\msg{\mathrm{message}}

\def\coh{\mathrm{coh}}
\def\cohi{\mathrm{\widetilde{coh}}}
\def\rel{\mathrm{rel}}
\def\fold{\mathit{fold}\,}

\def\restrC{\accentset{C}{\xi}}
\def\restrT{\accentset{T}{\xi}}
\def\restrS{\accentset{S}{\xi}^p_i}
\def\restrW{\accentset{W}{\xi}^p_i}

\def\ctx{\mathit{ctx}}
\def\codom{\mathrm{codomain}}
\def\maybe{\mathrm{Maybe\,}}


\subsection{Solving N-Queens Problem with Weak Commitment Search}
\label{sec:N-Queens-WCS}

A good example of a problem, that is solved much faster
using \emph{Weak-commitment search}, is \emph{n-queen} problem.
In the problem description, given in chapter \ref{chapter:csp}, queens' positions
are defined by a pairs $\left<x,y\right>$, that are single variables. In order to
use Weak-commitment search, the position variables should be separated.
On figure \ref{fig:4-Queens} is presented a step-by-step solution of
4-Queens problem.

% % % % % % % % % % % % % % % % % % % % % % % % % % % % % % % % % % % % % % % %


\newcommand{\mkBoard}{
  \def\range{1,...,\BoardSize}
  \foreach \i in \range
  \foreach \j in \range
    { \pgfmathsetmacro\diff{int(\i-\j)}
      \ifthenelse{\isodd{\diff}}
        {\def\currentNodeStyle{\NodeStyleOdd}}
        {\def\currentNodeStyle{\NodeStyleEven}}
      \node[\currentNodeStyle] at (\i,\j) {};
    }
  \ifthenelse{\isundefined{\ShowBoardHeader}} % show headers
             {}{
              \foreach \i in \range
                \node[\ShowBoardHeader] at ($(\i,\BoardSize+1)$) {\i};
              \foreach \i in \range
                \node[\ShowBoardHeader] at (0,\i) {\AlphAlph{\i}};
             }
}

\def\drawQueen[#1][#2](#3,#4){%
  \ifthenelse{\isundefined{\ShowQueenName}}
    { \node[#1] (#2) at (#3, #4) {\QueenSymb}; }
    { \node[#1, label={[opacity=.2,\ShowQueenName]center:\QueenSymb}]
           (#2) at (#3, #4) {#2}; }
}

\def\queenMoves[#1][#2](#3,#4){%
  \drawQueen[#1][#2](#3,#4)%
  % Horizontal
  \draw[-, \MoveStyle, #1](0.5,#4) -- ($(#3-0.4, #4)$);%
  \draw[-, \MoveStyle, #1]($(#3+0.4, #4)$) -- ($(\BoardSize+0.5,#4)$);%
  % Vertical
  \draw[-, \MoveStyle, #1](#3, 0.5) -- ($(#3, #4-0.4)$);%
  \draw[-, \MoveStyle, #1]($(#3, #4+0.4)$) -- ($(#3,\BoardSize+0.5)$);%
  % Diagonal 1
  \pgfmathsetmacro\diagUp{int(#4-#3)}
  \pgfmathsetmacro\diagUpLY{max(\diagUp,0)}
  \pgfmathsetmacro\diagUpLX{\diagUpLY-\diagUp}
  \pgfmathsetmacro\diagUpRY{min(\BoardSize,\BoardSize+\diagUp)}
  \pgfmathsetmacro\diagUpRX{\diagUpRY-\diagUp}
  \draw[-, \MoveStyle, #1]($(\diagUpLX+0.5, \diagUpLY+0.5)$)%
                       -- ($(#3-0.4, #4-0.4)$);%
  \draw[-, \MoveStyle, #1]($(#3+0.4, #4+0.4)$)%
                       -- ($(\diagUpRX+0.5,\diagUpRY+0.5)$);%
  % Diagonal 2
  \pgfmathsetmacro\diagDown{int(#3+#4-1)}
  \pgfmathsetmacro\diagDownY{min(\diagDown,\BoardSize)}
  \pgfmathsetmacro\diagDownX{\diagDown-\diagDownY}
  \draw[-, \MoveStyle, #1]($(\diagDownX+0.5, \diagDownY+0.5)$)%
                       -- ($(#3-0.4, #4+0.4)$);%
  \draw[-, \MoveStyle, #1]($(#3+0.4, #4-0.4)$)%
                       -- ($(\diagDownY+0.5,\diagDownX+0.5)$);%
}


% 1: list: name/style/priority/show_moves/x/y
\newcommand{\agentsBoard}[2][]{
  \agentsOnBoard{#2}
  \agentsBoardTable{#2}{#1}
}

\newcommand{\agentsOnBoard}[1]{
  \mkBoard
  \foreach \name/\style/\priority/\moves/\x/\y in {#1}
    {\ifthenelse{\equal{\y}{x}}{} % \or{\equal{\y}{x}}{\equal{\x}{x}}
      {\ifthenelse{\boolean{\moves}}
                  { \queenMoves[\style][\name](\x,\y) }
                  { \drawQueen [\style][\name](\x,\y) }}}
}

\newcommand{\agentsBoardTable}[2]{
  \gdef\mrows{}
  \foreach \name/\style/\priority/\moves/\x/\y in {#1}
    { \ifthenelse{\equal{\y}{x}}
                 {\gdef\ty{$\times$}}
                 {\gdef\ty{\AlphAlph{\y}}}
      \xappto{\mrows}{\priority\&\name\&\x\&\ty\noexpand\\}
    };

  \matrix[matrix of nodes, nodes in empty cells, anchor=north,
          ampersand replacement=\&,
          row sep=-\pgflinewidth, column sep=-\pgflinewidth,
          nodes={draw=none, text depth=0.1ex, text height=1.2ex, align=center},
          row 1/.style={nodes={draw=none, inner sep=0pt, text depth=1ex, font=\bf}},
          column 1/.style={minimum width=1.7cm},
          column 2/.style={minimum width=1cm},
          column 3/.style={minimum width=0.5cm},
          column 4/.style={minimum width=0.5cm},
          #2
  ] (AT) at ($(\BoardSize/2+0.5,0.2)$) {Priority\&Name\&x\&y\\\mrows};
}


\newcommand{\selectRow}[3][]{
  \draw[rounded corners, #1]
      (#2-0.3,0.6) rectangle node {#3}
      (#2+0.3,\BoardSize+0.4);
}

\newcommand{\selectPartialSolution}[2][]{
  \node[#1, fit=(AT-2-3) (AT-#2-4)]{};
}

\newcommand{\badPartialSolution}[1]{
  \selectPartialSolution[draw, red, thick, inner sep=0pt]{#1} % , label={[red]right:BAD}
}

\def\newBadPosition(#1,#2){
  \node[draw=red, circle, dotted, line width=5pt, inner sep=8pt] at (#1,#2) {};
}

\def\badPosition(#1,#2){
  \newBadPosition(#1,#2)
  \node[fill=red, circle, inner sep=5pt] at (#1,#2) {};
}

% \tikzset{badY/.style={row #1 column 4/.style={nodes={color=red}}}}

\newcommand{\QueensPicture}[2][]{
  \begin{tikzpicture}[
    oddCellStyle/.style={draw, minimum size=1cm},
    evenCellStyle/.style={oddCellStyle, fill=gray!50},
    queenMove/.style={line width=3pt, line cap=round, shorten <=1pt, shorten >=2pt,
                      darkgray},
    boldText/.style={font=\bf},
    badY/.style={row #1 column 4/.style={nodes={color=red}}},
    myGreen/.style = {green!50!black},
    #1
    ]
  \def\NodeStyleOdd{oddCellStyle}
  \def\NodeStyleEven{evenCellStyle}
  \def\BoardSize{4}
  \def\QueenSymb{\BlackQueenOnWhite}
  \def\MoveStyle{queenMove}
  \def\ShowBoardHeader{boldText}
  \def\ShowQueenName{}

  #2

  \end{tikzpicture}
}


\newcounter{subfiguregroup}
\setcounter{subfiguregroup}{0}
\renewcommand\thesubfigure{\Roman{subfiguregroup}-\arabic{subfigure}}
\def\nextSubfigGroup{%
  \stepcounter{subfiguregroup}%
  \setcounter{subfigure}{0}%
}

% \crule{0.9}
\bigskip
\begin{figure}[H]
  \captionsetup{singlelinecheck=off,indention=10pt, parskip=10pt}
  \caption[Step-by-step solution of 4-Queens]
    {
      Example of 4-Queens problem solution, using modified Weak-commitment search.

    The modification consists in distribution of the variables between
    queen-representing agents. Each queen agent would now control two separated
    variables: $x$ and $y$ with domains $\overline{1,N}$, where $N$ is chessboard size.
    Each queen would also receive \emph{priority}, that establishes its relations
    with other agents.

    The agents assign variables (one-by-one) in the order, established by their
    priorities. When an agent cannot find valid variable assignation:
    \begin{enumerate}
      \item The agent notifies its \emph{supervisor}: the agent with the following priory.
            If the notification is sent by \emph{root} agent (with highest priority),
            then it means that no solution exists.
      \item Bad \emph{partial solution} is guarded in \emph{generic manner} and then
            avoided by all the agents. The guarded bad solution ignores what agents
            have assigned the values: such positions configurations must be avoided,
            it is irrelevant which agents occupy the positions.
    \end{enumerate}
    The solution is found when all the agents managed to assign values to their
    variables.
    }
  \label{fig:4-Queens}
\end{figure}
\begin{enumerate}[I]
  % % % % % Step I % % % % %
  \item Agents set values of the first variable $x$ (ignoring all restrictions
        that make use of $y$) in the order of priority and fix it.
    \begin{figure}[H]\ContinuedFloat\nextSubfigGroup
      \centering
      \showSteps[column 4/.style={text opacity=0}]{0.21}
        {Q1/boldText/4/false/1/1,
         Q2/boldText/3/false/2/1,
         Q3/boldText/2/false/3/1,
         Q4/boldText/1/false/4/1}{}
    \end{figure}
  % % % % % Step II % % % % %
  \item Agents set values of the second variable $y$ in the order of priority.
        First two agents manage to find position assignation, but there is no
        value assignment for $Q3$. Therefore, current positions of $Q1$ and $Q2$
        (current \emph{partial solution}) are guarded as \emph{bad} and are avoided
        in the future.
    \begin{figure}[H]\ContinuedFloat\nextSubfigGroup
      \centering
      \showSteps{0.21}
        {Q1/boldText/4/true/1/1,
         Q2/boldText/3/true/2/3}{}
      \queensSubfig{0.21}{
        \agentsBoard[row 4 column 4/.style={nodes={color=red}}]
          {Q1/boldText/4/true/1/1,
           Q2/boldText/3/true/2/3,
           Q3/boldText/2//3/x}
        \selectRow[red, thick]{3}{\textbf{Q3}}
        \newBadPosition(2,3)
        \badPartialSolution{3}
      }
    \end{figure}
  % % % % % Step III % % % % %
  \item $Q2$ is forced to move, because it's direct subordinate $Q3$ failed to
        assign any value. After that $Q3$ manages to find a place for it, but
        not $Q4$. Bad configuration is guarded.
    \begin{figure}[H]\ContinuedFloat\nextSubfigGroup
      \centering
      \showSteps{0.21}
        {Q1/boldText/4/true/1/1,
         Q2/boldText/3/true/2/4,
         Q3/boldText/2/true/3/2}
        {\badPosition(2,3)}
      \queensSubfig{0.21}{
        \agentsBoard[row 5 column 4/.style={nodes={color=red}}]
          {Q1/boldText/4/true/1/1,
           Q2/boldText/3/true/2/4,
           Q3/boldText/2/true/3/2,
           Q4/boldText/1//4/x}
        \badPosition(2,3)
        \selectRow[red, thick]{4}{\textbf{Q4}}
        \newBadPosition(3,2)
        \badPartialSolution{4}
      }
    \end{figure}
  % % % % % Step IV % % % % %
  \item $Q3$ is forced to move, but there is no valid $y$ coordinate for it.
  % % % % % Step V % % % % %
  \item $Q2$ is forced to move, but there is no place for it neither.
    \begin{figure}[H]\ContinuedFloat\nextSubfigGroup
      \centering
      \queensSubfig{0.21}{
        \agentsBoard{Q1/boldText/4/true/1/1,
                     Q2/boldText/3/true/2/4}
        \badPosition(2,3)
        \badPosition(3,2)
      }
      \queensSubfig{0.21}{
        \agentsBoard[row 4 column 4/.style={nodes={color=red}}]
          {Q1/boldText/4/true/1/1,
           Q2/boldText/3/true/2/4,
           Q3/boldText/2/false/3/x}
        \badPosition(2,3)
        \badPosition(3,2)
        \selectRow[red, thick]{3}{}
        \newBadPosition(2,4)
        \badPartialSolution{3}
      }
    \nextSubfigGroup
      \queensSubfig{0.21}{\agentsBoard{Q1/boldText/4/true/1/1}
                          \badPosition(2,3) \badPosition(2,4)}
      \queensSubfig{0.21}{
        \agentsBoard[row 3 column 4/.style={nodes={color=red}}]
          {Q1/boldText/4/true/1/1,
           Q2/boldText/3/true/2/x}
        \badPosition(2,3)
        \badPosition(2,4)
        \selectRow[red, thick]{2}{}
        \newBadPosition(1,1)
        \badPartialSolution{2}
      }
    \end{figure}
  % % % % % Step VI % % % % %
  \item $Q1$ is forced to move on the next tile.
        Then $Q2$, $Q3$ and $Q4$ occupy the only valid $y$ positions sequentially.
        Now when all the agents managed to assign all the
        variables, the solution is found.
    \begin{figure}[H]\ContinuedFloat\nextSubfigGroup
      \showSteps{0.21}
        {Q1/boldText/4/true/1/2,
         Q2/boldText/3/true/2/4,
         Q3/boldText/2/true/3/1}
        {\badPosition(1,1)}
      \queensSubfig{0.21}{
        \agentsBoard
          {Q1/boldText/4/true/1/2,
           Q2/boldText/3/true/2/4,
           Q3/boldText/2/true/3/1,
           Q4/boldText/1/true/4/3}
        \badPosition(1,1)
        \selectPartialSolution[draw, thick, myGreen, inner sep=0pt]{5}
      }
    \end{figure}
\end{enumerate}

\end{document}
