\documentclass[../header]{subfiles}

\begin{document}

\providecommand{\rootdir}{..}

\def\domain{\mathrm{domain}}

\def\domain{\mathrm{domain}}
\def\pop{\mathrm{pop}}

\def\behaviour{\mathrm{behaviour}}
\def\act{\mathrm{act}}
\def\react{\mathrm{react}}
\def\state{\mathrm{state}}
\def\action{\mathrm{action}}
\def\msg{\mathrm{message}}

\def\coh{\mathrm{coh}}
\def\cohi{\mathrm{\widetilde{coh}}}
\def\rel{\mathrm{rel}}
\def\fold{\mathit{fold}\,}

\def\restrC{\accentset{C}{\xi}}
\def\restrT{\accentset{T}{\xi}}
\def\restrS{\accentset{S}{\xi}^p_i}
\def\restrW{\accentset{W}{\xi}^p_i}

\def\ctx{\mathit{ctx}}
\def\codom{\mathrm{codomain}}
\def\maybe{\mathrm{Maybe\,}}



\section{Solving CSPs with Multiagent Systems}
\label{sec:CSP-Agents}

Multiagent Systems (MAS) have been successfully used to solve
different \emph{constraints satisfaction problems} \cite{MAS, MAS-Survey}.
Each problem variable is given to an agent, that would try to set its value,
avoiding contradictions with the values of its ``neighbouring'' agents.
There are different methods and techniques for implementing agents behavior.
In \cite{MAS-Survey} the MAS are classified by two properties of the agents,
that form the system: heterogeneity and communications.
The classification is presented in table \ref{table:MAS-Table}. It denotes
that CSPs are usually solved by \emph{heterogeneous communicating} agents.

\begin{table}
  \centering
  \begin{tabular}{P{0.1\textwidth} P{0.3\textwidth}|P{0.4\textwidth}}
    & \centering Homogeneous & Heterogeneous \\
    \begin{minipage}{0.1\textwidth}
      \rotatebox[origin=tr]{90}{Non-Communicating}
    \end{minipage}
                      & \begin{minipage}{0.3\textwidth}
                        \footnotesize \begin{itemize}[leftmargin=5px, rightmargin=0px]
                          \item Reactive vs. deliberative agents
                          \item Local or global perspective
                          \item Modeling of other agents' states
                          \item How to affect others
                        \end{itemize}\end{minipage}
                      & \begin{minipage}{0.4\textwidth}
                        \footnotesize\begin{itemize}[leftmargin=10px, rightmargin=0px]
                          \item[]
                          \item Benevolence vs. competitiveness
                          \item Fixed vs. learning agents (arms race, credit assignment)
                          \item Modeling of others’ goals, actions, and knowledge
                          \item Resource management (interdependent actions)
                          \item Social conventions
                          \item Roles
                          \item[]
                        \end{itemize}\end{minipage}
                      \\\hline
    \begin{minipage}{0.1\textwidth}
      \rotatebox[origin=tr]{90}{Communicating}
    \end{minipage}
                      & \begin{minipage}{0.3\textwidth}
                        \footnotesize\begin{itemize}[leftmargin=5px, rightmargin=0px]
                          \item Distributed sensing
                          \item Communication content
                        \end{itemize}\end{minipage}
                      & \begin{minipage}{0.4\textwidth}
                        \footnotesize\begin{itemize}[leftmargin=10px, rightmargin=0px]
                          \item[]
                          \item Understanding each other
                          \item Planning communicative acts
                          \item Benevolence vs. competitiveness
                          \item Negotiation
                          \item Resource management (schedule coordination)
                          \item Commitment/decommitment
                          \item Collaborative localization
                          \item Changing shape and size
                        \end{itemize}\end{minipage}
  \end{tabular}
  \caption[Issues arising in the various scenarios as reflected in the literature]
          {Issues arising in the various scenarios as reflected in the literature
          \cite[Table~2]{MAS-Survey}.}
  \label{table:MAS-Table}
\end{table}




\bigskip
One of the simplest methods consists in taking all possible variable domain values
and eliminating the values, that are in contradiction with the neighbors.
The process results in a solution when all the agents have exactly one
possible value; or it fails if any of the agents runs out of the possible
values to propose. A natural question arises: in what order should the agents
prune their domains?
It can be resolved by assigning a distinct comparable priority value to each agents.
In such case a value should be excluded only if neighbor's priority exceeds own priority.
Combined with \emph{backtracking} technique, it guarantees finding a solution,
if it exists.

\noindent
An agent $a_i$
\begin{enumerate}
  \item controls variable $V_i$ with a finite $\domain~V_i =
        \{{\dot v}^i_j\}_{j=1}^{N_i}$.
  \item has sets of possible ${\tilde V}_i$ and excluded ${\tilde V}_i^-$ values,
  \item has priority $p$.
\end{enumerate}

\noindent
In the beginning each agent assigns
\begin{align*}
  &{\tilde V}_i   &&\leftarrow \domain~v_i       && \textit{(any domain value is possible)} & \\
  &{\tilde V}_i^- &&\leftarrow \{\}              && \textit{(no excluded values at start)}  &\\
  &V_i            &&\leftarrow \pop~{\tilde V}_i && \textit{(pop first possible value and assign it to the variable)} &\\
\end{align*}

Every time current value $V_i$ is assigned, an agent must notify its
neighbors about it with a \emph{proposal} message, containing the value.
The interactions between agents are heavily based on their priority and are
defined for pairs, where one of the agents always has a higher priority.

\bigskip\noindent
Agents communications can be divided in two groups, each depending on the relative
priorities.
\begin{enumerate}
  \item Testing constraints of the \textbf{lower priority} agents by the
    \textbf{higher priority} ones.
    Each agent must notify its neighbors on every value assignment and wait
    for a confirmation from the higher priority ones. If any of them responds
    negatively, the agent must try next \emph{possible} value from ${\tilde V}_i$.
    In case that possible values have ended (${\tilde V}_i = \emptyset$), the agent
    must notify its direct superior (the neighbor with the next highest priority)
    about it, starting the \emph{backtracking}.
  \item Backtracking permits to exit dead-end solution branches, thus allowing
    to check \emph{every possible values combination} with limited resources.

    As written in previous item, the backtracking process starts when an agent
    finds itself without possible values to assign. It has searched all of its
    domain, but no solution to its part of a problem exists. Therefore it is
    necessary that a higher priority agent changes its value. It is better to
    keep the process ordered and the next priority agent is a rational choice
    for a supervisor.

    Upon a notification about solution incoherence, the supervisor
    must try to change its value $V_i$ and add the old one to the
    excluded set ${\tilde V}_i^-$.
    If neither this agent has more possible values to assign, it should propagate
    backtracking request to its own supervisor, and so on. Otherwise, it should
    demand its subordinates to reset their ${\tilde V}_*$ and $V_*$ states
    before changing its value. New \emph{possible values} sets are formed, considering
    the excluded values: ${\tilde V}_i = \domain~v_i \setminus {\tilde V}_i^-$.
    Backtracking request by the \emph{top priority} agent means that no solution
    could be found.
\end{enumerate}

\medskip\noindent
The \textbf{stop criteria} is the lack of negotiation activity, that signalizes
that all the values assignations have been confirmed, thus a solution, satisfying
the constraints, was found.

\medskip\noindent
Any set of \underline{consistent} (that are not in conflict)
 variables assignations is called \emph{partial solution}.
$$ \lbrace\left<V_i,{\dot v}^i_*\right>\rbrace_{i \in I} $$

\medskip\noindent
A \underline{complete solution} is a partial solution,
that contains assignations for all the problem variables.

\bigskip
%%%%%%%%%%%%%%%%%%%%%%%%%%%%%%%%%%%%%%%%%%%%%%%%%%%%%%%%%%%%%%%%%%%%%%%%%%%%%%%%
There are various extensions of the method that improve its speed.
One of such techniques is \emph{Weak-commitment Search}\cite{Weak-commitment},
that removes the drawbacks of backtracking algorithms. It utilizes a min-conflict
heuristic, and therefore is compared by the authors with min-conflict backtracking.
Both algorithms \underline{add variables} to the \emph{partial solution} one by one.
The difference is in algorithms' \emph{commitment} to current partial solution.
Backtracking never abandons its p. solution until proven \emph{hopeless}:
it tries every possible combination of variables assignations from ${\tilde V}_i$
for each variable $V_i$, that isn't within the solution yet.
\emph{Weak commitment}, as it follows from the name, commits \emph{weakly}:
whole partial solution is abandoned if even one variable $V_i$ has no valid
assignment. Furthermore, any failed partial solution is guarded and avoided in
the future. The resulting difference is that Weak-commitment algorithm revises
bad decision (as backtracking does) without exhaustive search, that speeds it up
and still guarantees algorithm completeness (if any solution exists, it would be found).


\end{document}
