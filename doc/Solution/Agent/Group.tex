\documentclass[../../ThesisDoc]{subfiles}

\begin{document}

\providecommand{\rootdir}{../..}

\def\domain{\mathrm{domain}}

\def\domain{\mathrm{domain}}
\def\pop{\mathrm{pop}}

\def\behaviour{\mathrm{behaviour}}
\def\act{\mathrm{act}}
\def\react{\mathrm{react}}
\def\state{\mathrm{state}}
\def\action{\mathrm{action}}
\def\msg{\mathrm{message}}

\def\coh{\mathrm{coh}}
\def\cohi{\mathrm{\widetilde{coh}}}
\def\rel{\mathrm{rel}}
\def\fold{\mathit{fold}\,}

\def\restrC{\accentset{C}{\xi}}
\def\restrT{\accentset{T}{\xi}}
\def\restrS{\accentset{S}{\xi}^p_i}
\def\restrW{\accentset{W}{\xi}^p_i}

\def\ctx{\mathit{ctx}}
\def\codom{\mathrm{codomain}}
\def\maybe{\mathrm{Maybe\,}}


%%%%%%%%%%%%%%%%%%%%%%%%%%%%%%%%%%%%%%%%%%%%%%%%%%%%%%%%%%%%%%%%%%%%%%%%%%%%%%%%
\subsection{Group Agents}

\emph{Group agents} play active role in the negotiation.
Initially, each group generates a \emph{random candidate}.
Generation process ensures that \emph{class coherence} constraints
($\restrC$ and $\restrT$) are satisfied, thus implementing \textit{Capabilities}
and \textit{Time Consistency} contexts at once. It consists of three stages:
\begin{enumerate}
  \item \emph{Class-Cores Pool} generation.
  \item Generation of a \emph{class-cores} from the \emph{pool}.
  \item \emph{Day-Time-Room} assignment.
  %  Values are generated randomly for the
  %       entire set of class-cores, respecting time restrictions $\restrT$.
  %       % Classes duration is specified alongside each discipline.
\end{enumerate}

\noindent
After estimating generated candidate's coherence, a decision is chosen:\\
\begin{itemize}
  \item Create new class-core pool.
  \item Try next class-core.
  \item Try another day-time-room configuration.
  \item Assess coherence in another \emph{mode}.
  \item Negotiate over a candidate.
  \item Fail with error.
\end{itemize}


%%%%%%%%%%%%%%%%%%%%%%%%%%%%%%%%%%%%%%%%%%%%%%%%%%%%%%%%%%%%%%%%%%%%%%%%%%%%%%%%
\subsubsection{Class-Cores}
A class core is considered to be the immutable part of a class, that doesn't
change during its existence. It links \emph{group} and \emph{professor} agents
through a \emph{discipline}, respecting the \emph{capability} restrictions $\restrC$.
It can also be seen as an ``abstract'' class, with no time or classroom assigned.

A \emph{Class-Cores Pool} is a lazy random sequence of \emph{class-cores},
that contains a class for each discipline, needed by the group.

\emph{Pool} creation requires knowledge of the existing professor agents with
the disciplines, that the represented person can teach.
\begin{enumerate}
  \item For each discipline needed select professors, that can teach it.
        Randomize professors lists.
  \item Lazily generate all possible combinations \emph{professor} -- \emph{discipline}.
  \item When getting next combination, assign the generating \emph{group}.
\end{enumerate}

%%%%%%%%%%%%%%%%%%%%%%%%%%%%%%%%%%%%%%%%%%%%%%%%%%%%%%%%%%%%%%%%%%%%%%%%%%%%%%%%
\subsubsection{Day -- Time -- Room}

In order to obtain concrete classes, the cores must be completed with day, room,
beginning time and duration (end time), while respecting both discipline duration
requirements and time consistency $\restrT$.

Assigning the missing values can be seen as mapping (randomly) the set of cores onto
3-dimensional space: $\left< \text{day, time, room} \right>$.
In order to do it, a random generator is used. It uses a history variable to
store generated values and test time consistency of the new ones against the former.
The following process is repeated for each class core (using same generator).
\begin{enumerate}
  \item Generate random \emph{day}.
  \item Select random \emph{classroom}.
  \item Generate \emph{beginning time}.
  \item Get class duration from the \emph{discipline}, contained in argument class-core.
        Calculate \emph{end time}.
  \item Test \emph{end time} consistency (upper bound).
  \item Test \emph{time consistency} of the values from (1-4) using the history.
  \item If new values are consistent, add them to history, assign to class-core
        and return it.
        Otherwise, repeat from (1).

\end{enumerate}
After the assignment has been done for all the class-cores in the given set,
the history variable is reset.

\medskip

With rather big set of classrooms and adequate time discretization, the generated
values can be considered unique with a rather high probability.
\red{Put numbers here?}


%%%%%%%%%%%%%%%%%%%%%%%%%%%%%%%%%%%%%%%%%%%%%%%%%%%%%%%%%%%%%%%%%%%%%%%%%%%%%%%%

\subsubsection{Decisions}

The decision is made, over a \emph{candidate}, that is first assessed by agent's contexts
(section \ref{sec:solution-contexts}). The decision might also depend on the
\emph{best candidate} and other \emph{state variables}.
Regardless of the decision taken, candidate's coherence is compared with the one
of current best candidate (if there is none, 0 is used). In case new candidate's
coherence was found greater, then the best candidate variable is updated.


\secpartl{Negotiate over a candidate}
\noindent
The central decision, it should be taken when possible. Starts negotiation process
with a goal to modify the candidate to suite external restrictions.
It is properly discussed in section \ref{sec:c-negotiation}.


\secpartl{New class-core pool}
\noindent
Because the pool already contains all possible professor -- discipline combinations,
this decision is only taken if new professor counterparts have appeared
in the negotiation.

\secpartl{Try next class core}
\noindent
This decision is the only way to change the three class parameters
defined by the core (group, professor, discipline), because these values
cannot be changed by a negotiation, as it defines its participants.
It is taken if a negotiation over a candidate failed.

\secpartl{Try another day -- time -- room assignment}
\noindent
This decision causes result, similar to a negotiation --- reassignment of
the day, time and room values of all the classes of a candidate, but without
grantee of coherence, other than the common one (class and time consistency).
It is used for quick random search of a better starting candidate variant to begin a
negotiation over.

\secpartl{Assess coherence in another mode}
\noindent
A mode, as already written in section \ref{sec:solution-contexts}, is an extra
argument, used at candidate's assessment. It is passed by the contexts to every
evaluated relation.

Current implementation uses two assessment modes:
\emph{preliminary} and \emph{final}. The former one should disable certain
relations, it is intended primarily to be used by professor agents. It is the
default assessment mode.
In order for a candidate to become a \emph{partial solution} to the schedule
problem, it should be assessed in \emph{final} mode.

\secpartl{Fail with error}
\noindent
Terminate the agent and report given error as a result.


% % % % % % % % % % % % % % % % % % % % % % % % % % % % % % % % % % % % % % % %

% % % % % % % % % % % % % % % % % % % % % % % % % % % % % % % % % % % % % % % %

\newcommand{\drawAllCandidates}[2][]{
  \newpicture{
    \drawFirstCandidate{#2}{false}
    \drawSecondCandidate{#2}{false}
    \drawThirdCandidate{#2}{false}
  }{#1}
}

\newcommand{\drawTwoCandidates}[2][]{
  \newpicture{
    \drawFirstCandidate{#2}{false}
    \drawSecondCandidate{#2}{false}
  }{#1}
}

\newcommand{\drawFirstAlone} [2][]{ \newpicture{\drawFirstCandidate{#2}{true}}{#1} }
\newcommand{\drawSecondAlone}[2][]{ \newpicture{\drawSecondCandidate{#2}{true}}{#1} }
\newcommand{\drawThirdAlone} [2][]{ \newpicture{\drawThirdCandidate{#2}{true}}{#1} }

% % % % % % % % % % % % % % % % % % % % % % % % % % % % % % % % % % % % % % % %

\newcommand{\newpicture}[2]{
  \begin{tikzpicture}
  [
    agent/.style={draw, minimum size=0.7cm, inner sep=1},
    group/.style={agent, diamond},
    prof/.style={agent, circle},
    class/.style={draw, circle, fill},
    #2
  ]#1\end{tikzpicture}
}

%% Candidate #1 (by Group 1) with professors 1-4.
%% 1: r, 2: alone flag
\newcommand{\drawFirstCandidate}[2]{
  \def\r{#1}
  \def\alone{#2}

  \node[group, blue] (G1) {$G_1$};
  \draw (45 :\r)  node[prof] (P1) {$P_1$};
  \draw (0  :\r)  node[prof] (P2) {$P_2$};
  \draw (-45:\r)  node[prof] (P3) {$P_3$};
  \draw (-90:\r)  node[prof] (P4) {$P_4$};
  \draw [-, blue] (G1) -- (P1) node [midway, class, blue] {};
  \draw [-, blue] (G1) -- (P2) node [midway, class, blue] {};
  \draw [-, blue] (G1) -- (P3) node [midway, class, blue] {};
  \draw [-, blue] (G1) -- (P4) node [midway, class, blue] {};
  \ifthenelse{\boolean{\alone}}{}{
    \node[draw, ellipse, blue, dotted, fit=(G1) (P1) (P2) (P3) (P4),
          label={[blue]above:$\mathrm{Candidate}_1^{i_1}$},
          inner sep=-9pt, xshift=-12pt ] {};
    }
}

%% Candidate #2 (by Group 2) with professors 1,2,5,6.
%% 1: r, 2: alone flag
\newcommand{\drawSecondCandidate}[2]{
  \def\r{#1}
  \def\alone{#2}

  \ifthenelse{\boolean{\alone}} {
    \def\xshift{0pt}
    \def\yshift{0pt}
    }{
    \def\xshift{7cm}
    \def\yshift{3cm}
    }

  \begin{scope}[xshift=\xshift, yshift=\yshift]
    \node[group, red] (G2) {$G_2$};
    \ifthenelse{\boolean{\alone}}{
      \draw (180 :\r)  node[prof] (P1) {$P_1$};
      \draw (225 :\r)  node[prof] (P2) {$P_2$};
      }{}
    \draw (-90:\r) node[prof] (P5) {$P_5$};
    \draw (0  :\r) node[prof] (P6) {$P_6$};
    \draw [-, red] (G2) -- (P1) node [midway, class, red] {};
    \draw [-, red] (G2) -- (P2) node [midway, class, red] {};
    \draw [-, red] (G2) -- (P5) node [midway, class, red] {};
    \draw [-, red] (G2) -- (P6) node [midway, class, red] {};
    \ifthenelse{\boolean{\alone}}{}{
      \node[draw, ellipse, red, dotted, fit=(G2) (P1) (P2) (P5) (P6),
            label={[red]above:$\mathrm{Candidate}_2^{i_2}$},
            inner sep=-10pt, rotate=-6, xshift=19pt, yshift=13pt
            ] {};
      }
  \end{scope}
}


%% Candidate #3 (by Group 3) with professors 2,3,5,7.
%% 1: r, 2: alone flag
\newcommand{\drawThirdCandidate}[2]{
  \def\r{#1}
  \def\alone{#2}

  \ifthenelse{\boolean{\alone}} {
    \def\xshift{0pt}
    \def\yshift{0pt}
    }{
    \def\xshift{6cm}
    \def\yshift{-4.5cm}
    }

  \begin{scope}[xshift=\xshift, yshift=\yshift]
    \def\myGreen{green!40!black}
    \node[group, \myGreen] (G3) {$G_3$};
    \ifthenelse{\boolean{\alone}}{
      \draw (120 :\r)  node[prof] (P2) {$P_2$};
      \draw (225 :\r)  node[prof] (P3) {$P_3$};
      \draw (80  :\r)  node[prof] (P5) {$P_5$};
      }{}
    \draw (0:\r) node[prof] (P7) {$P_7$};
    \draw [-, \myGreen] (G3) -- (P2) node [midway, class, \myGreen] {};
    \draw [-, \myGreen] (G3) -- (P3) node [midway, class, \myGreen] {};
    \draw [-, \myGreen] (G3) -- (P5) node [midway, class, \myGreen] {};
    \draw [-, \myGreen] (G3) -- (P7) node [midway, class, \myGreen] {};
    \ifthenelse{\boolean{\alone}}{}{
      \node[draw, ellipse, \myGreen, dotted, fit=(G3) (P2) (P3) (P5) (P7),
            label={[\myGreen]right:$\mathrm{Candidate}_3^{i_3}$},
            inner sep=-20pt, rotate=-22, xshift=1pt, yshift=-10pt
            ] {};
      }
  \end{scope}
}

% % % % % % % % % % % % % % % % % % % % % % % % % % % % % % % % % % % % % % % %


\begin{figure}[b]
  \begin{subfigure}{0.3\textwidth}
    \drawFirstAlone{2cm} % using CandidateNegotiation.tikz
    \caption{}
    \label{fig:candidate-1}
  \end{subfigure}
  ~
  \begin{subfigure}{0.3\textwidth}
    \drawSecondAlone[trim left, xshift=1.7cm]{2cm} % using CandidateNegotiation.tikz
    \caption{}
    \label{fig:candidate-2}
  \end{subfigure}
  ~
  \begin{subfigure}{0.3\textwidth}
    \drawThirdAlone{2cm} % using CandidateNegotiation.tikz
    \caption{}
    \label{fig:candidate-3}
  \end{subfigure}

  \caption{Possible candidates, generated by some \emph{group} agents
            $G_1$ (\ref{fig:candidate-1}), $G_2$ (\ref{fig:candidate-2}) and
            $G_3$ (\ref{fig:candidate-3}).}
  \label{fig:candidates-alone}
\end{figure}

\end{document}
