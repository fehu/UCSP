\documentclass[../ThesisDoc]{subfiles}

\begin{document}

\providecommand{\rootdir}{..}

\def\domain{\mathrm{domain}}

\def\domain{\mathrm{domain}}
\def\pop{\mathrm{pop}}

\def\behaviour{\mathrm{behaviour}}
\def\act{\mathrm{act}}
\def\react{\mathrm{react}}
\def\state{\mathrm{state}}
\def\action{\mathrm{action}}
\def\msg{\mathrm{message}}

\def\coh{\mathrm{coh}}
\def\cohi{\mathrm{\widetilde{coh}}}
\def\rel{\mathrm{rel}}
\def\fold{\mathit{fold}\,}

\def\restrC{\accentset{C}{\xi}}
\def\restrT{\accentset{T}{\xi}}
\def\restrS{\accentset{S}{\xi}^p_i}
\def\restrW{\accentset{W}{\xi}^p_i}

\def\ctx{\mathit{ctx}}
\def\codom{\mathrm{codomain}}
\def\maybe{\mathrm{Maybe\,}}



\section{Formal definition}

\noindent
Formally speaking, a CSP is defined by its \emph{variables} $V$ with the
corresponding \emph{domains} and the \emph{constraints} $\{\xi\}$
over values assignation \cite{MAS}.

\begin{align*}
  V                &= \{v_i\}_{i=1}^N
& \{{\dot v}^i_j\} &= \domain(v_i)
& \xi              &: \{{\dot v}^i_\ast\}_{i=1}^N \mapsto [0,1]
\end{align*}


A variable defines a ``slot'' that can hold a value from the corresponding domain.
A solution to CSP is an assignation of the values ${\dot v}^i_\ast$
to the variables $V$, such that all the restriction hold.
\begin{equation}
  {\dot V} = \{{\dot v}^i_\ast\}_{i=1}^N \text{is a solution}
   \iff \forall \xi \in \{\xi\} \Rightarrow \xi({\dot V}) = 1
\end{equation}

The constraints above are defined in the most generalized form,
over the entire solution. There is a particular case, that is found in many problems
--- \emph{binary constraints}, imposed on \emph{pairs} of values:
$$\xi_2 : \left< {\dot v}^i_\ast, {\dot v}^j_\ast \right> \mapsto \{0,1\}$$

If all constrains are binary, than the satisfaction condition is
$$\forall \xi_2     \in \{\xi\},~
  \forall v_i, v_j  \in V | v_i \not= v_j
~ \Rightarrow ~ \xi_2(v_i, v_j) = 1
$$

\medskip

In the graph coloring problem, the variables are the \emph{colors to be assigned}
for each graph node $\{n_i\}_{i=1}^N$. In this case, all the variables have
the same domain values: four colors, for example \textit{Red}, \textit{Green},
\textit{Blue} and \textit{Yellow}. The constraint is binary and depends on graph
structure:
$$ \xi_2(x,y)= \begin{cases}
  0 & \mbox{if } \exists \text{~edge~} x \leftrightarrow y
                ~\land \text{~color~} x = \text{color~} y \\
  1 & \text{otherwise}
\end{cases}
$$


For the n-queens problem, the variables are queens' positions ---
pairs $\left< x,y \right>$, where $x$ is queen's horizontal position and
$y$ is the vertical one. The restrictions can be gathered within a single
constraint function:
$$\xi_2(\left<x_1,y_1\right>, \left<x_2,y_2\right>) =
    \begin{cases}
      0 & \mbox{if } \begin{cases}
                        &      x_1 = x_2  \lor y_1 = y_2 \\
                        \lor~& x_1 = y_2 \land x_2 = y_1 \\
                        \lor~& x_1-y_1 = x_2-y_2\\
                        \lor~& x_1+y_1 = x_2+y_2
                     \end{cases} \\
     1 & \text{otherwise}
    \end{cases}
$$

\begin{figure}
  \begin{subfigure}[b]{0.3\textwidth}
    \centering
  	\subfile{\rootdir/img/NQueens/Restrictions.tikz}
  	\caption{Queen moves/attacks.}
  	\label{fig:QueenMoves}
  \end{subfigure}
  \hfill
  \begin{subfigure}[b]{0.3\textwidth}
    \centering
  	\includegraphics[trim=100 260 370 370, clip] % width=\textwidth
               {\rootdir/img/easteuro}
  	\caption{Part of Central Europe \cite{UN-CEU-Map}.}
  	\label{}
  \end{subfigure}
  \hfill
  \begin{subfigure}[b]{0.3\textwidth}
  	\centering
    \subfile{\rootdir/img/ColorMap/CEUMapGraph.tikz}
  	\caption{Graph coloring problem.}
  	\label{fig:ColoringGraph}
  \end{subfigure}
  \caption{CSP examples}
\end{figure}

  In our \emph{university classes scheduling} problem (UCSP) the variables
are personal schedules, called \emph{timetables} in this thesis,
for each \emph{participant}: professor and group/student.
  A timetable consists of the day-time slots, where classes can be put.
  A \emph{class} is formed for teaching a group a specific \emph{subject},
connecting the participants (of each kind) and establishing a classroom and
the beginning--end \emph{time}.
  Problem \emph{constraints} can be divided into:
\begin{enumerate}
  \item \underline{Class constraints}: a class should be a productive event, so
    a professor should be able to \emph{teach} the class, the classroom should
    have the \emph{capacity} to hold all the students and be properly \emph{equipped},
    and the group should be \emph{enrolled} to class subject
    (further called \emph{discipline}).
  \item \underline{Time constraints}: no participant can have two classes
    at the same time (or intersecting in time), as mentioned before.
  \item \underline{Strong restrictions}: the restrictions, put on a participant, that
    \emph{must} be respected. May include working hours, fixed lunch recess time,
    and any other institution or person specific \emph{obligations}.
\end{enumerate}

\bigskip

%%%%%%%%%%%%%%%%%%%%%%%%%%%%%%%%%%%%%%%%%%%%%%%%%%%%%%%%%%%%%%%%%%%%%%%%%%%%%%%%
  Until now we were speaking about restrictions satisfaction, but the problem can
be extended to an \emph{optimization} one by defining an
\emph{objective function} over the values configurations.

  Unlike the restrictions (that yield boolean result), the objective functions
should have continuous codomains:
$$\tilde\xi : \{{\dot v}^i_\ast\}_{i=1}^N \mapsto \Re$$
  In order to facilitate the calculations and erase the internal differences of
the objectives, it's often required that the functions must be normalized:
the co-domains are restricted to $[-1,1]$ or $[0,1]$ intervals.

  For example, in case of our UCSP,
one can add \underline{personal preferences} or some institution criteria as
optimization parameters.
  Compliance with the preferences raises \emph{solution quality},
but it is assumed that the all of them cannot be fully met for all the
participants, because \emph{personal} preferences are often
contradictory for different persons.

\end{document}
