\documentclass[../ThesisDoc]{subfiles}
\begin{document}

\providecommand{\rootdir}{..}
\providecommand{\seccmd}[1]{\section{#1}}

An internal context requires no knowledge from other agents.
The combined coherence of all the internal contexts is called
\emph{inner coherence} and is denoted as $\cohi$.

% % % % % % % % % % % % % % % % % % % % % % % % % % % % % % % % % % % % % % % %
\seccmd{Obligations}

The obligations determine custom \emph{strong restrictions} over the classes.
As in the case of \emph{capabilities}, the obligation relations must yield
boolean result and make use of \emph{mode} relation argument.

In order to be coherent, a candidate must comply with \emph{all} the restrictions.

\medskip

Possible \emph{obligation relations} examples:
maximum and minimum classes per week, maximum classes per day,
lunch recess time, lower/upper class time limit, two classes must/cannot follow etc.

% \red{At the moment there are no obligations used (but they are supported).}

% % % % % % % % % % % % % % % % % % % % % % % % % % % % % % % % % % % % % % % %
\seccmd{Preferences}

The preferences define \emph{weak restrictions}. The relations values might be any
value within $[0,1]$ interval. To avoid overrestrictions, this context's \emph{threshold}
should decrease with time. Relations results are combined using \emph{multiplication}.
\emph{Initial threshold value} and \emph{threshold decrease rate} must be
provided for this context.

% \red{Possible preferences}


\end{document}
