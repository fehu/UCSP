\def\domain{\mathrm{domain}}

\section{Constraint Satisfaction Problems}

There are many problems than require positioning or assigning something,
respecting established \emph{restrictions}. \emph{Graph coloring} and
\emph{n-queen} chess problem are classical constraint satisfaction problems (CSPs).

The graph coloring problem comes from cartography, where it was needed to color
countries on political maps, in such a way that no country had a land border with
a country of the same color. It was found that \textbf{any} map can be colored
with only four colors.

The n-queen problem is known in chess as \emph{eight queen puzzle}.
Queens in chess can move/attack to/at any square,
that is in a the same row, column or diagonal with the queen.
In the puzzle one needs to place eight (the size of a chess desk)
queens on the desk, so that none of them is threatened by another.
N-queens problems is a generalization of that puzzle, where $n$ queens need
to be placed on a $n \times n$ desk.

Constraint satisfaction problems are found in many areas:
machine vision, natural language processing, theorem proving,
planning and in our problem --- scheduling \cite{MAS}.

\bigskip
%%%%%%%%%%%%%%%%%%%%%%%%%%%%%%%%%%%%%%%%%%%%%%%%%%%%%%%%%%%%%%%%%%%%%%%%%%%%%%%%
\noindent
Formally speaking, a CSP is defined by its \emph{variables} $V$ with the
corresponding \emph{domains} and the \emph{constraints} $\{\xi\}$
over values assignation \cite{MAS}.

\begin{align*}
  V                &= \{v_i\}_{i=1}^N
& \{{\dot v}^i_j\} &= \domain(v_i)
& \xi              &: \{{\dot v}^i_\ast\}_{i=1}^N \mapsto [0,1]
\end{align*}


A variable defines a ``slot'' that can hold a value from the corresponding domain.
A solution to CSP is an assignation of the values ${\dot v}^i_\ast$
to the variables $V$, such that all the restriction hold.
\begin{equation}
  {\dot V} = \{{\dot v}^i_\ast\}_{i=1}^N \text{is a solution}
   \iff \forall \xi \in \{\xi\} \Rightarrow \xi({\dot V}) = 1
\end{equation}

The constraints above are defined in the most generalized form,
over the entire solution. There is a particular case, that is found in many problems
--- \emph{binary constraints}, imposed on \emph{pairs} of values:
$$\xi_2 : \left< {\dot v}^i_\ast, {\dot v}^j_\ast \right> \mapsto [0,1]$$

If all constrains are binary, than the satisfaction condition is
$$\forall \xi_2     \in \{\xi\},~
  \forall v_i, v_j  \in V | v_i \not= v_j
~ \Rightarrow ~ \xi_2(v_i, v_j) = 1
$$

\medskip

In the graph coloring problem, the variables are the \emph{colors to be assigned}
for each graph node $\{n_i\}_{i=1}^N$. In this case, all the variables have
the same domain values: four colors, for example \textit{Red}, \textit{Green},
\textit{Blue} and \textit{Yellow}. The constraint is binary and depends on graph
structure:
$$ \xi_2(x,y)= \begin{cases}
  0 & \mbox{if } \exists \text{~edge~} x \leftrightarrow y
                ~\mathit{AND} \text{~color~} x = \text{color~} y \\
  1 & \text{otherwise}
\end{cases}
$$

% For the n-queens problem, there are two ways of representing the variables.One
% can see queens' positions as a single variable --- pairs $\left< x,y \right>$.
% But it's better to handle them separately, as it would be shown in \ref{TODO}.
% In this case each queen would have two variables:
% horizontal and vertical positions $x$ and $y$.

For the n-queens problem, the variables are queens' positions ---
pairs $\left< x,y \right>$, where $x$ is queen's horizontal position and
$y$ is the vertical one. The restrictions can be gathered within a single
constraint function:
$$\xi_2(\left<x_1,y_1\right>, \left<x_2,y_2\right>) =
    \begin{cases}
      0 & \mbox{if } \begin{cases}
                        &      x_1 = x_2  \lor y_1 = y_2 \\
                        \lor~& x_1 = y_2 \land x_2 = y_1 \\
                        \lor~& |x_1-y_1| = |x_2-y_2|
                     \end{cases} \\
     1 & \text{otherwise}
    \end{cases}
$$

\todo\red{: needs to be rewritten after the introduction is done}

In our \emph{university classes scheduling} problem (UCSP) the variables
are personal schedules, called \emph{timetables} in this thesis,
for each professor, group/student and classroom. A timetable consists
of the day-time slots, where \emph{classes} can be put.
Problem \emph{constraints} can be divided into:
\begin{enumerate}
  \item \underline{Class constraints}: a class should be a productive event, so
    a professor should be able to \emph{teach} the class, the classroom should
    have the \emph{capacity} to hold all the students and be properly \emph{equipped},
    and the group should be \emph{inscribed} to class subject
    (further called \emph{discipline}).
  \item \underline{Time constraints}: no participant can have two classes
    at the same time (or intersecting in time), as mentioned before.
  \item \underline{Strong restrictions}: the restrictions, put on a participant, that
    \emph{must} be respected. May include working hours, fixed lunch recess time,
    and any other institution or person specific \emph{obligations}.
  \item \red{\underline{Weak restrictions} (MOVE)}: the restrictions, that \emph{should} be respected,
    but are not critical to the solution. Compliance with theese restrictions
    raises \emph{solution quality}, but it is assumed that the all of them
    cannot be fully met for all the participants --- they are intended to
    represent personal \emph{preferences}.
\end{enumerate}

\bigskip
Until now we were speaking only about restrictions, but the problem can
be extended to an \emph{optimization} one. For example, in case of our UCSP,
one can add \emph{personal preferences} or some institution criteria as
optimization parameters. \todo\red{: here come Weak restrictions}




\bigskip
%%%%%%%%%%%%%%%%%%%%%%%%%%%%%%%%%%%%%%%%%%%%%%%%%%%%%%%%%%%%%%%%%%%%%%%%%%%%%%%%
\noindent
Solving these problems usually presents difficulties due to the amount of
possible combinations to consider for a solution.

\todo

\bigskip
%%%%%%%%%%%%%%%%%%%%%%%%%%%%%%%%%%%%%%%%%%%%%%%%%%%%%%%%%%%%%%%%%%%%%%%%%%%%%%%%
\noindent

The researchers have been looking for means of solving such
problems within reasonable time.
During the last years different algorithms and techniques where developed,
such as
\emph{dynamic constraint satisfaction based on extension particle swarm
      optimization algorithm} \cite{CSPswarm},
\emph{dynamic state bounding} \cite{CSPdynStateBound},
\emph{conflict-vector detection} \cite{CSPtimetable},
\emph{neural networks} \cite{CSPneuro},
\emph{ant colony optimization} \cite{CSPcunningACO, CSPlimmemACO},
\emph{selective hyper-heuristics} \cite{CSPhypHeur}
and \emph{agents} \cite{CSPagent2013, CSPagent2014, DCSPagent1998}.


The \emph{agent negotiation} approach is usually used for solving distributed
CSPs (DCSPs) \cite{DCSPagent1998, DCSP2013, CSPagent2014}.
In this case the constraints are \emph{distributed} among the agents instead of
being gathered in one place.

\todo \red{: write something about each technique?}

\medskip

Those methods do not seek for the optimal solutions, focusing instead on
``rather good'' ones, that can be obtained within reasonable time using admissible
computing resources.
