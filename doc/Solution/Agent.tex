\documentclass[../ThesisDoc]{subfiles}

\begin{document}

\providecommand{\rootdir}{..}

\def\domain{\mathrm{domain}}

\def\domain{\mathrm{domain}}
\def\pop{\mathrm{pop}}

\def\behaviour{\mathrm{behaviour}}
\def\act{\mathrm{act}}
\def\react{\mathrm{react}}
\def\state{\mathrm{state}}
\def\action{\mathrm{action}}
\def\msg{\mathrm{message}}

\def\coh{\mathrm{coh}}
\def\cohi{\mathrm{\widetilde{coh}}}
\def\rel{\mathrm{rel}}
\def\fold{\mathit{fold}\,}

\def\restrC{\accentset{C}{\xi}}
\def\restrT{\accentset{T}{\xi}}
\def\restrS{\accentset{S}{\xi}^p_i}
\def\restrW{\accentset{W}{\xi}^p_i}

\def\ctx{\mathit{ctx}}
\def\codom{\mathrm{codomain}}
\def\maybe{\mathrm{Maybe\,}}


\subsection{Agent Implementation}

%%%%%%%%%%%%%%%%%%%%%%%%%%%%%%%%%%%%%%%%%%%%%%%%%%%%%%%%%%%%%%%%%%%%%%%%%%%%%%%%

As it says in section \ref{sec:NegotiatingAgents}, an agent is defined by its
\emph{behavior}, that is composed from \emph{proactive} and \emph{reactive}
behavior functions. Both functions use agent's \emph{state} (that should include
agent's \emph{contexts}).

Current agent implementation extends the definition above by adding:
\begin{enumerate*}[1)]
  \item unique agent name for disambiguous referencing;
  \item \emph{execution state} for execution control;
  \item \emph{result variable} for collecting results.
\end{enumerate*}

$$ a = \left< \mathit{name},
              \behaviour^a,
              \state,
              \state_\mathit{exec},
              \state_\mathit{result},
       \right> $$

%%%%%%%%%%%%%%%%%%%%%%%%%%%%%%%%%%%%%%%%%%%%%%%%%%%%%%%%%%%%%%%%%%%%%%%%%%%%%%%%
\noindent
\emph{Execution state} and \emph{result variable} are the only two entities,
accessible both to the agent and external environment. The internal state,
denoted as $\state$ is accessible only to the agent itself (e.g. its behavior functions).


The implementation uses three execution states: \texttt{run}, \texttt{pause}
and \texttt{terminate}. Newly created agents are set to \texttt{pause}.
\texttt{Pause} state prevents \emph{proactive} behavior execution until the state is changed.
Message handling (\emph{reactive} behavior) is unaffected.
Setting \texttt{Run} state initiates repetitive $\behaviour\act$ execution.
\texttt{Terminate} state forces both behavior aspects to stop (irreversibly).

Agent can report results using \emph{result variable}.
Reporting result doesn't stop an agent.
There can be two type of results: the values reported by an agent or
the descriptions of errors, that occurred during agent's execution
and forced it to stop.

\medskip

\noindent
In order to create a new agent, one needs to specify: \\
\begin{tabularx}{\textwidth}{X c}
  \multicolumn{1}{m{0.4\textwidth}}{
    \begin{enumerate}
      \item unique name,
      \item initial internal state,
      \item behavior functions,
      \item result type.
    \end{enumerate}
  }
&  (A)
\end{tabularx}

\bigskip

\noindent
Agents are implemented using \emph{Software Transactional Memory} (STM)
\cite{STMCode07} --- a promising concurrency paradigm.
They are executed in two computational threads:
one for message handling and another for proactive action.

%%%%%%%%%%%%%%%%%%%%%%%%%%%%%%%%%%%%%%%%%%%%%%%%%%%%%%%%%%%%%%%%%%%%%%%%%%%%%%%%
\subsubsection{References}
As agent's internal state must be private, an agent should not be accessible directly.
In order to hide both the state and behavior implementation, \emph{agent reference}
is used instead. It provides four ways to interact with an agent:
\begin{enumerate}
  \item Execution management: run, pause, terminate.
  \item Result recollection: gets the result, reported by an agent, if any.
        There is also a modification, that \emph{waits} (blocks thread) for an agent
        to report a result (or terminate with an error).
  \item \emph{Send} a message. No response is expected. Handled by
        \emph{reactive behavior} function.
  \item \emph{Ask} a question. The expected \emph{response} is defined by
        the question message sent. Asking an agent returns a \emph{promise of response},
        that can be either handled asynchronously or awaited (blocking).
\end{enumerate}

Of course a reference also provides agent's \emph{unique name}. Two
reference are considered to be appointing to the same agent if the names
of the underlying agents match.


%%%%%%%%%%%%%%%%%%%%%%%%%%%%%%%%%%%%%%%%%%%%%%%%%%%%%%%%%%%%%%%%%%%%%%%%%%%%%%%%
\subsubsection{Roles}

As it is stated in section \ref{sec:NegotiatingAgents}, an \emph{agent role}
generalizes agents behavior and serves as a template for new agents.
A role defines both \emph{types} and the \emph{template} for the agents, associated with it.

The types defined by it are:
  \begin{enumerate*}[1)]
    \item internal state,
    \item result,
    \item template argument.
  \end{enumerate*}
The last type should contain the information, needed for \emph{individual} agent creation.

The template, defined by a role, provides all the information needed to create agents (A),
given the corresponding \emph{template argument}.


%%%%%%%%%%%%%%%%%%%%%%%%%%%%%%%%%%%%%%%%%%%%%%%%%%%%%%%%%%%%%%%%%%%%%%%%%%%%%%%%
\subsubsection{Agent System}

\emph{Agent system} controls agents creation, execution and termination
(both successful and not).
It helps to mass-manage agents by
\begin{enumerate*}[1)]
  \item registering all the agents created by the system,
  \item allowing to search references by name (and role),
  \item providing methods for bulk execution state changes and result recollection.
\end{enumerate*}


Agent system makes extensive use of agent roles. Internally, agents,
registered in a system, are grouped by roles.
Apart from defining \emph{types} of corresponding agents' states and results,
a \emph{role} defines templates for agents creation, alongside the types, expected
as arguments for the tempate functions.
A system uses these templates to create new agents.

\end{document}
