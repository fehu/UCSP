%\documentclass{article}


%\usepackage{xcolor, amsmath}
%\newcommand{\red}[1]{{\color{red} #1}}


\def\coh{\mathrm{coh}}


%\begin{document}

\section{Problem Formalization}

Let \begin{itemize}
\item $D=\{d_i\}$ be the set of \emph{disciplines}.
  A discipline may be seen as class descriptor, it contains
  academic program name and information about special requirements,
  such as laboratories.
\item $G=\{g_i\}$ be the set of \emph{groups}.
  A group unites some students. In this thesis it is assumed that
  \textbf{each student belongs strictly to one group}.
  A group has a set of disciplines, that it is obliged to take by an
  academical program.
\item $P=\{p_i\}$ be the set of \emph{professors}.
  Each professor can teach a set of disciplines, that is determined
  by the institution. There are two kinds of professors:
  \emph{full-time} and \emph{part-time}. The difference is that the
  latter have more flexible obligations, while the former have preference
  in classes assignment.
\item $R=\{r_i\}$ be the set of \emph{classrooms}.
  A classroom has two properties: capacity and special equipment installed.
\item $D=\{\bar d_i\}$ be the set of working \emph{days}.
\item $T=\{t_i\}$ be \emph{discrete time} (limited by working hours).
\end{itemize}
\medskip

A \emph{class} denotes an event \emph{discipline} for some \emph{group}
at some specific \emph{time interval} in the denoted \emph{classroom},
performed by some \emph{professor}.

$$ c \sim \left< d, g, p, r, \bar d, t, t \right> $$
\medskip

A \emph{candidate} to solution is a configuration (set) of classes.
It can be assessed by it's \emph{coherence}, that is properly discussed
in \ref{sectionCoherence}. Coherence is estimated within an agent and
depends on agent's knowledge.

\begin{align*}
 \tilde{c} &= \{c_i\}  &\coh[a]: \tilde c &\mapsto \Re
\end{align*}


The coherence function is designed in such a way, that given a candidate,
each agent, mentioned by candidate's underlying classes, yields the same
coherence assessment for the candidate.


\begin{align}
  \label{eq:coh-fun-independ}
  \begin{aligned}
    &\forall \tilde{c} = \{c_k\} \\
    &\forall c_i \sim \left< \dots, g_i, p_i, r_i, \dots \right> \\
    &\forall c_j \sim \left< \dots, g_j, p_j, r_j, \dots \right>
  \end{aligned}
& \implies
  \begin{aligned}
   \coh[g_i](\tilde c) &= \coh[g_j](\tilde c) = \\
   = \coh[p_i](\tilde c) &= \coh[p_j](\tilde c) = \\
   = \coh[r_i](\tilde c) &= \coh[r_j](\tilde c)
  \end{aligned}
\end{align}

\medskip

\noindent
The coherence function defines an order over the candidates and permits
agents to extract the \emph{acceptable} ones using a threshold.
The \textbf{negotiation goal} is to encounter an \emph{acceptable candidate}
for each negotiation participant.






%\end{document}
