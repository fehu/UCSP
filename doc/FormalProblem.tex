\documentclass[ThesisDoc]{subfiles}
\begin{document}

\section{Problem Formalization}
\label{sec:ProblemFormal}
\green{Definir el problema del "scheduling" como CSP}



Let \begin{itemize}
\item $D=\{d_i\}$ be the set of \emph{disciplines}.
  A discipline may be seen as class descriptor, it contains
  academic program name and information about special requirements,
  such as laboratory equipment.
\item $G=\{g_i\}$ be the set of \emph{groups}.
  A group unites some students. In this thesis it is assumed that
  \textbf{each student belongs strictly to one group}.
  A group has a set of disciplines, that it is obliged to take by an
  academical program.
\item $P=\{p_i\}$ be the set of \emph{professors}.
  Each professor can teach a set of disciplines, that is determined
  by the institution. There are two kinds of professors:
  \emph{full-time} and \emph{part-time}. The difference is that the
  latter have more flexible obligations, while the former have preference
  in classes assignment.
\item $R=\{r_i\}$ be the set of \emph{classrooms}.
  A classroom has two properties: capacity and special equipment installed.
\item $D=\{\bar d_i\}$ be the set of working \emph{days}.
\item $T=\{t_i\}$ be \emph{discrete time} (limited by working hours).
\item $ c \sim \left< d, g, p, r, \bar d, t_b, t_e \right> $ be a \emph{class}
  of discipline $d$ for group $g$, taught by professor $p$, that takes place
  at classroom $r$ every day $d$ in time interval $t_b$--$t_e$.
\item $\{\restrC\}$ be the restrictions over the classes.
      $\restrC : c \mapsto \mathrm{Bool}$
\item $\restrT$ be the \emph{time-consistency} restriction, that ensures
  classes non-intersection for each participant.
      $\restrT : c \mapsto c \mapsto \mathrm{Bool}$
\item $\{\restrS\}$ be the rest of strong restrictions, or \emph{obligations},
      of participant $p$.
      $\restrS : \{c\} \mapsto \mathrm{Bool}$
\item $\{\restrW\}$ be the weak restrictions, or \emph{preferences}, of participant $p$.
      In order to avoid overrestrictions, the preferences should weaken with time.
      $\restrW : \tau \mapsto \{c\} \mapsto (0,1]$
% \item $\xi^p$ be the combined restrictions of agent $p$.
\item $\tau$ be \emph{negotiation time}. $\tau \in \mathbb{N}$
\end{itemize}
\medskip

A \emph{candidate} to solution $\tilde{c}$ is a configuration (set) of classes
$\{c\}$, that form participant's individual schedule. An \emph{acceptable candidate}
respects all the restrictions $\{\xi^p_i\}$ (of the participant $p$),
thus can be selected as the solution (individual).

\bigskip

\noindent
A solution to the scheduling problem is the set of \emph{all} the individual solutions.

\end{document}
