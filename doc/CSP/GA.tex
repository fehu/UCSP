\documentclass[../header]{subfiles}

\begin{document}

\providecommand{\rootdir}{..}

\def\domain{\mathrm{domain}}

\def\domain{\mathrm{domain}}
\def\pop{\mathrm{pop}}

\def\behaviour{\mathrm{behaviour}}
\def\act{\mathrm{act}}
\def\react{\mathrm{react}}
\def\state{\mathrm{state}}
\def\action{\mathrm{action}}
\def\msg{\mathrm{message}}

\def\coh{\mathrm{coh}}
\def\cohi{\mathrm{\widetilde{coh}}}
\def\rel{\mathrm{rel}}
\def\fold{\mathit{fold}\,}

\def\restrC{\accentset{C}{\xi}}
\def\restrT{\accentset{T}{\xi}}
\def\restrS{\accentset{S}{\xi}^p_i}
\def\restrW{\accentset{W}{\xi}^p_i}

\def\ctx{\mathit{ctx}}
\def\codom{\mathrm{codomain}}
\def\maybe{\mathrm{Maybe\,}}



\subsection{Genetic Algorithm}
GA is an evolutionary computational method for solving optimization problems,
including constraints satisfaction. The method is inspired by \emph{evolution
theory}, its genetic part to be concrete. It is known that living organisms
store information about themselves and transmit (part of) it to its offspring.
In case of sexual reproduction, each participating sex gives its share of
genetic information, thus ``mixing'' parents properties in the offspring.
It's called \emph{crossover}.
Crossover can create combinations of the parents, but not create something new.
The latter is provided by \emph{mutations} --- random changes in the genome, due
to copying errors. Because of mutation's randomness, the results cannot be known
in advance, and such event might both better or worsen the ``victim''. Mutations
are relatively rare in the nature, thus maintaining species populations more or
less alike.

But if the mutations are rare, then how are the new properties fixed within
populations? The response to this isn't found in the genetics, but rather in
organism's \emph{adaptation} to the environment. A property that gives an
\emph{advantage} before the rest of the population
\begin{enumerate*}[1)]
  \item raises organism's chances of survival, extending mean life length and
        potential number of children within its lifetime;
  \item raises organism's ``value'' in the eyes of potential mate, thus further
        increasing potential size of its offspring.
\end{enumerate*}
With time, the property is spread throughout the population, because of the
advantages it gives.

\bigskip\noindent
Genetic algorithms search for better \emph{adopted} ``organisms'', described
by their genetic code --- chromosomes, using \emph{crossover} and \emph{mutations}
as its \emph{genetic operations}.
The chromosomes are \underline{sequences of \emph{genes}} --- variables with finite
domains. The degree of \emph{adaptation} is calculated by
\underline{\emph{fitness function}}.

GAs are iterative algorithms, working with \emph{populations}, that consist of
\emph{chromosomes}, representing possible solutions. The algorithms search
for better \emph{fitted} solutions, trying to maximize mean population
\emph{fitness} on each iteration.
Notable advantage of genetic algorithms is that they are capable of
``getting out'' \underline{local} maximums due to the mutations, thus finding
the \underline{global} maximum, provided adequate configuration.

\begin{enumerate}[start=0]
  \item Create initial population (randomly).
  \item\label{list:GA:start} Evaluate \emph{fitness} of each chromosome in the population.
  \item If \emph{stop criterion} applies, return best chromosomes and terminate.
  \item\label{list:GA:xover} Select best chromosomes and apply \emph{crossover}.
  \item\label{list:GA:mutate} Select chromosomes for mutation and \emph{mutate} them.
  \item Create new population from the crossover children (\ref{list:GA:xover}),
        mutated chromosomes (\ref{list:GA:mutate}) and the original population
        (\ref{list:GA:start}).
        Go to \ref{list:GA:start}.
\end{enumerate}

\end{document}
