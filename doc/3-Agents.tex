\documentclass[ThesisDoc]{subfiles}

\providecommand{\rootdir}{.}

\def\domain{\mathrm{domain}}

\def\domain{\mathrm{domain}}
\def\pop{\mathrm{pop}}

\def\behaviour{\mathrm{behaviour}}
\def\act{\mathrm{act}}
\def\react{\mathrm{react}}
\def\state{\mathrm{state}}
\def\action{\mathrm{action}}
\def\msg{\mathrm{message}}

\def\coh{\mathrm{coh}}
\def\cohi{\mathrm{\widetilde{coh}}}
\def\rel{\mathrm{rel}}
\def\fold{\mathit{fold}\,}

\def\restrC{\accentset{C}{\xi}}
\def\restrT{\accentset{T}{\xi}}
\def\restrS{\accentset{S}{\xi}^p_i}
\def\restrW{\accentset{W}{\xi}^p_i}

\def\ctx{\mathit{ctx}}
\def\codom{\mathrm{codomain}}
\def\maybe{\mathrm{Maybe\,}}


\begin{document}

\chapter{Agents}

An \emph{agent} has no generally accepted definition, but the idea is traced back
to the antique times. The survey of \cite[sec.~2.2]{PNoriega} gives several
definitions, proposed by different authors, ranging from philosophy to AI.

The notion of agent appears in Aristotle's works:
 ``entity that acts with a purpose, within a social context''.
The prætorian roman law defined an agent as
``a person who acts on behalf of a principal for
a specific purpose and under limited delegation of authority and responsibility''.

The earliest use of the term agent in AI was
``a program that is capable of executing an action vicariously''.
Later it was formulated as \emph{a computer system, which}
\begin{enumerate}
  \item has a degree of autonomy in determining its behavior,
  \item interacts with humans and or other agents,
  \item perceives the environment and reacts to it, and
  \item exhibits a goal directed behavior.
\end{enumerate}

\bigskip

There are three general approaches to defining/describing agents:
\begin{enumerate}
  \item \emph{Agent Theories} discuss what an agent \emph{is} and formalize
    it in mathematical form.
  \item \emph{Agent architectures} deal with the \emph{implementations}
    of the Agent Theories, physical (hardware) and/or logical (software).
  \item \emph{Agent languages} are software systems, that allow communication
    between the agents (including human/living ones).
\end{enumerate}


Many authors have thought of agents as \emph{logical theories}, that perceive
the environment as formulas, that are processed within or against those theories.
An agent is usually required to be capable of proactive behavior, not
just responses to environment changes. Some authors impose stronger demands upon
the agents, such as mobility, truthfulness, benevolence, rationality.

The most notable agent theory is the \emph{Beliefs-Desires-Intentions} (BDI) one,
that deduces agent's intentions (and the following sequence of actions)
from its desires (goals) and an uncertain image of the environment (beliefs).


Some works (for example \cite{UAB-Thesis} and \cite{PNoriega}) propose
layered architectures, where each layer (for example: \emph{Beliefs} or \emph{Intentions})
has its own logic. The ``logics'' are then united using \emph{bridge rules}.

\medskip

\noindent
One could summarize that an agent is primarily a computational entity (software),
but it might have a (robotic) ``body''.
The BDI theory, for example, allows to abstract existence of a ``body'',
by seeing it as an actuator withing the environment, that \emph{tries} to
implement agent's current \emph{intention}.
The actual results of an interaction should not be taken as predefined, but rather
as probabilistic, observable through the \emph{beliefs}. As an example, one can
see a human being as an agent of its \emph{ego}, that determines the \emph{desires}.
As we all well know, not all the \emph{intention} always come out as expected.
It depends, of course, on how general one understands intention, but whether
it's an intention to go to Mars or to lift a pen, there is always a degree of
uncertainty. Our brain, as an actuator, transforms motion intentions into
neuronal signals, even if there destination was severed or the neural connections
were damaged; there is no way of knowing the results of intention implementation
except from new information.

\bigskip

\noindent
An agent should be capable of:
\begin{enumerate}
  \item \emph{autonomous} and \emph{goal directed} behavior,
  \item perception of and interaction with the environment,
  \item communication with other agents (including humans).
\end{enumerate}

%%%%%%%%%%%%%%%%%%%%%%%%%%%%%%%%%%%%%%%%%%%%%%%%%%%%%%%%%%%%%%%%%%%%%%%%%%%%%%%%
%%%%%%%%%%%%%%%%%%%%%%%%%%%%%%%%%%%%%%%%%%%%%%%%%%%%%%%%%%%%%%%%%%%%%%%%%%%%%%%%

\subfile{\rootdir/Agents/SolvingCSPs}
\subfile{\rootdir/Agents/NQueenWCS}

%%%%%%%%%%%%%%%%%%%%%%%%%%%%%%%%%%%%%%%%%%%%%%%%%%%%%%%%%%%%%%%%%%%%%%%%%%%%%%%%
%%%%%%%%%%%%%%%%%%%%%%%%%%%%%%%%%%%%%%%%%%%%%%%%%%%%%%%%%%%%%%%%%%%%%%%%%%%%%%%%



\section{Negotiating Agents}
%%%%%%%%%%%%%%%%%%%%%%%%%%%%%%%%%%%%%%%%%%%%%%%%%%%%%%%%%%%%%%%%%%%%%%%%%%%%%%%%
\label{sec:NegotiatingAgents}

A \emph{negotiation} is a process of communication between heterogeneous agents with a
goal of finding such a configuration of variables assignations, that all the
agents involved agree upon it. The results and the time of a negotiation depend
on agents \emph{cooperation}. The latter implies that agents must pursue a
\emph{common goal}, rather than be selfish; it can be formulated as
``the good of many outweigh the good of the one''.


\bigskip

\noindent
In this thesis the following definition of a class of agents is used:
a \textbf{negotiating agent} is an isolated proactive computational entity,
capable of sending and receiving messages.
The \emph{isolation} denotes that agents' variables are protected
from outside access; messaging is the only way an agent can be interacted with.
The \emph{pro-activity} implies a capacity of acting asynchronously,
with no ``external'' cause.


\medskip

An agent is defined by it's behaviour --- the combination of its
\emph{proactive} and \emph{reactive} (message handling) functions.

\begin{flalign*}
  &\behaviour = \left< \behaviour_\act, \behaviour_\react \right>\\
  &\behaviour_\act   : \state \mapsto \action \\
  &\behaviour_\react : \state \times \msg \mapsto \action
\end{flalign*}

Therefore two agents with same behaviour functions should be considered two instances
of the same agent. In must be noted, that all the diference in the behaviour of
two instances is produced by the differences in the states
(both agent's internal state and the environment's one).

\medskip

The agents, participating in a negotiation, are considered \emph{heterogeneous}
(to any degree: from complete heterogeneity to homogeneity).
In order to generalize some agents behaviour, \textbf{negotiation roles} are introduced.
A role describes whom or what an agents represents in the negotiation and
defines \emph{behaviour archetype} --- the rules to build
agent's \emph{behaviour functions}, given some \emph{role-specific} knowledge.

The agents must use a common \emph{communication protocol}, to ensure
understanding between agents of the same or different roles.


%%%%%%%%%%%%%%%%%%%%%%%%%%%%%%%%%%%%%%%%%%%%%%%%%%%%%%%%%%%%%%%%%%%%%%%%%%%%%%%%

\end{document}
