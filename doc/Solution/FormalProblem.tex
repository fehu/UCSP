\documentclass[../header]{subfiles}
\begin{document}

\providecommand{\rootdir}{..}


\def\domain{\mathrm{domain}}

\def\domain{\mathrm{domain}}
\def\pop{\mathrm{pop}}

\def\behaviour{\mathrm{behaviour}}
\def\act{\mathrm{act}}
\def\react{\mathrm{react}}
\def\state{\mathrm{state}}
\def\action{\mathrm{action}}
\def\msg{\mathrm{message}}

\def\coh{\mathrm{coh}}
\def\cohi{\mathrm{\widetilde{coh}}}
\def\rel{\mathrm{rel}}
\def\fold{\mathit{fold}\,}

\def\restrC{\accentset{C}{\xi}}
\def\restrT{\accentset{T}{\xi}}
\def\restrS{\accentset{S}{\xi}^p_i}
\def\restrW{\accentset{W}{\xi}^p_i}

\def\ctx{\mathit{ctx}}
\def\codom{\mathrm{codomain}}
\def\maybe{\mathrm{Maybe\,}}


\section{Problem Formalization}
\label{sec:ProblemFormal}


Let \begin{itemize}
\item $D=\{d_i\}$ be set of \emph{disciplines}.
  A discipline may be seen as class descriptor, it contains
  academic program name and information about special requirements,
  such as laboratory equipment.
\item $G=\{g_i\}$ be set of \emph{groups}.
  A group unites some students. In this thesis it is assumed that
  \textbf{each student belongs strictly to one group}.
  A group has a set of disciplines, that it is obliged to take by an
  academical program: $D^G_i$.
\item $P=\{p_i\}$ be set of \emph{professors}.
  Each professor can teach a set of disciplines, that is determined
  by the institution.
  % There are two kinds of professors:
  % \emph{full-time} and \emph{part-time}. The difference is that the
  % latter have more flexible obligations, while the former have preference
  % in classes assignment. All professors are treated in the same manner;
  % \emph{part-time} agents have stronger obligations ... \todo\red{: it should be
  % written elsewhere.}
\item $R=\{r_i\}$ be set of \emph{classrooms}.
  A classroom has two properties: capacity and special equipment installed.
\item $\bar D=\{\bar d_i\}$ be the set of working \emph{days}.
\item $\bar T=\{\bar t_i\}$ be \emph{discrete time} (limited by working hours).
\item $\mathrm{cc} \sim \left< d, g, p \right>$ be a \emph{class core}, that
      links group $g$ with professor $p$ through a \emph{possible class} for
      discipline $d$.
\item $ c \sim \left< \mathrm{cc}, r, \bar d, \bar t_b, \bar t_e \right> $
      be a \emph{class} --- an ``instance'' of some \emph{class core}, that has
      certain day, time (beginning \& end) and room assigned.
\item $\{\restrC\}$ be the restrictions over the classes.
      $\restrC : c \mapsto \Bool$
\item $\restrT$ be the \emph{time-consistency} restriction, that ensures
  classes non-intersection for each participant.
      $\restrT : c \mapsto c \mapsto \Bool$
\item $\{\restrS\}$ be the rest of strong restrictions, or \emph{obligations},
      of participant $p$.
      $\restrS : \{c\} \mapsto \Bool$
\item $\{\restrW\}$ be the weak restrictions, or \emph{preferences}, of participant $p$.
      In order to avoid overrestrictions, the preferences should weaken with time.
      $\restrW : \tau \mapsto \{c\} \mapsto (0,1]$
\item $\tau$ be \emph{negotiation time}. $\tau \in \mathbb{N}$
\end{itemize}
\medskip

\noindent
A \emph{candidate to partial solution} $\tilde{c}_a^k$ (just \emph{candidate} further)
is a set of classes $\{c_a^*\}$, that:
\begin{enumerate}
  \item Complies with \emph{class restrictions}: \\
    $$\forall c_a^* \in \tilde{c}_a^k \implies \restrC(c_a^*) = \true$$
  \item Complies with \emph{time-consistency} restriction: \\
    $$\forall c_a^*, c_a^\circ \in \tilde{c}_a^k ~|~ c_a^* \not= c_a^\circ \implies
      \restrT(c_a^*, c_a^\circ) = \true$$
  \item Is \textbf{proposed by a group}: $a = g_i$.
  \item Has enough classes for each discipline, that group $g_i$ is enrolled.
        Doesn't have excess classes or classes for other groups:
        $$\tilde{c}_i^k = \bigcup\limits_{d \in D^G_i}
                            \lbrace c_i^n \rbrace_d ~|~ \mathsmaller{
                                \mathrm{discipline}(c_i^n) = d;~
                                \sum\limits_n \mathrm{time}(c_i^n) =
                                              \mathrm{time}(d)   }$$

\end{enumerate}
\bigskip

\noindent
A solution to the scheduling problem is a valid union of \emph{all} partial solutions.
Solution space is presented on figure \ref{fig:ScheduleHypercube}.

\begin{figure}[H]
  \centering
  \resizebox{\textwidth}{!}{
    \subfile{\rootdir/img/ScheduleHypercube/GRPT.tikz}
  }
  \caption[Schedule 4-dimensions space]
          {Schedule 4-dimensions space. Each \emph{group} $G_i$,
          \emph{professor} $P_k$ and \emph{classroom} $R_j$ has a
          \underline{timetable} (Day $\times$ Time) of its own.}
  \label{fig:ScheduleHypercube}
\end{figure}

\end{document}
