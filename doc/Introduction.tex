\documentclass[ThesisDoc]{subfiles}
\begin{document}

\section{Introduction}

University \emph{class scheduling} problem (UCSP) is faced each academic term
by many institutes throughout the world. A sufficient number of classes must
be created and then assigned to the students and professors in such a way,
that persons' schedules don't intersect and the persons are (rather) satisfied
with the assignment. It is usually done manually.

While UCSP research has already started by 1960s \red{put cites?},
no accessible computational equipment had the resources to handle the problem
in general and/or for entire universities until the recent times.
With rapid computer performance growth in the 1990s, computationally
difficult problems could be researched with minor expenses and many
solution search methods were developed for various kinds of promlems.
Class scheduling is a \emph{constraint satisfaction problem} (CSP) ---
a large kind of problems, that is properly discussed in section \ref{sec:csp}.

\red{
  A number of UCSP solutions were proposed recently [find cites], but \dots
}

\medskip

\noindent
This \thisdoc proposes an extendable UCSP solution with support for
\emph{schedule personalization}. The latter is achieved by using agent
negotiations between the representatives of each \emph{students/groups} and
\emph{professors}. The representing agents permit defining
\emph{personal preferences} and \emph{restrictions}, that are used to optimize
the schedules.

\medskip

\noindent
A solution for general UCSP problem
(formalized in section \ref{sec:ProblemFormal}) \red{was implemented}
(see section \ref{sec:solution}) \red{and tested} (see section \ref{sec:test}).
Possible improvements are discussed in section \ref{sec:improve}.



\end{document}
