\documentclass[../ThesisDoc]{subfiles}

\begin{document}

\providecommand{\rootdir}{..}

\def\Re{\mathbb{R}}

\def\domain{\mathrm{domain}}

\def\domain{\mathrm{domain}}
\def\pop{\mathrm{pop}}

\def\behaviour{\mathrm{behaviour}}
\def\act{\mathrm{act}}
\def\react{\mathrm{react}}
\def\state{\mathrm{state}}
\def\action{\mathrm{action}}
\def\msg{\mathrm{message}}

\def\coh{\mathrm{coh}}
\def\cohi{\mathrm{\widetilde{coh}}}
\def\rel{\mathrm{rel}}
\def\fold{\mathit{fold}\,}

\def\restrC{\accentset{C}{\xi}}
\def\restrT{\accentset{T}{\xi}}
\def\restrS{\accentset{S}{\xi}^p_i}
\def\restrW{\accentset{W}{\xi}^p_i}

\def\ctx{\mathit{ctx}}
\def\codom{\mathrm{codomain}}
\def\maybe{\mathrm{Maybe\,}}

\def\TRUE{\mathit{TRUE}}


\section{Contexts}
\label{sec:solution-contexts}


A \emph{context} represents an aspect for optimization/restriction, considered
by an agent. It defines context-specific information and relations over the
information graph, with the corresponding combination functions.
Each context defines some \emph{Details}, that describe coherence evaluation.
There are two types of relations: \emph{binary} and over the \emph{whole graph}.
The former ones are applied to every pair of nodes and the results are combined
by folding\footnote{
  In functional programming, fold (or reduce) is a family of higher order
  functions that process a data structure in some order and build a return value.
  Typically, a fold deals with two things: a combining
  function\textsuperscript{\ref{footnote:binary-operation}},
  and a data structure, typically a list of elements.
  The fold then proceeds to combine elements of the data structure using
  the function in some systematic way. (\url{https://wiki.haskell.org/Fold})
} using an \emph{associative binary operation}\footnote{%
  Operator $\bullet$ is associative
  $\Leftrightarrow (a \bullet b) \bullet c = a \bullet (b \bullet c)$
    \label{footnote:binary-operation}
} $f^\cup_2$ (for each relation).
The latter, as follows from the name, are defined over whole information graph,
thus permitting more flexible evaluation.

Coherence values, assessed by all the relations, are combined using another
\emph{associative binary function} $f^\cup$.
The result is \emph{coherence at the context}.

\begin{align}
\begin{split}
  \coh[\mathrm{context}] &: \mathit{Candidate} \mapsto \Re \times \mathit{Details} \\
  \coh[\mathrm{context}] &= \fold(f^\cup) \circ \mathrm{assess}
\end{split}
\end{align}
\emph{where}
\begin{align*}
    \mathrm{assess}(c) &=
      \lbrace \coh[\rel](c) ~|~ \rel \in \mathrm{relations}(\mathrm{context})
      \rbrace\\
    \coh[\rel](c) &= \begin{cases}
                       \rel(c) & \textbf{if } \rel \text{ is \emph{over whole graph}}
                       \\
                       \fold (f^\cup_2[\rel]) ~
                                (\lbrace \rel(x,y) ~|~ x,y \in c, ~ x \not= y
                                 \rbrace)
                          & \textbf{if } \rel \text{ is \emph{binary}}
                     \end{cases}
\end{align*}
\begin{flalign*}
  f^\cup, f^\cup_2 &: (\Re \times \mathit{Details}) \times (\Re \times \mathit{Details})
                      \mapsto \Re \times \mathit{Details} & &
\end{flalign*}
\begin{align*}
  \rel &:\lbrace \mathit{Information\,piece} \rbrace
              \mapsto \Re \times \mathit{Details}
       & \textbf{if } & \rel \text{ is \emph{over whole graph}}
  \\
  \rel &:\mathit{Information\,piece} \times \mathit{Information\,piece}
              \mapsto \Re \times \mathit{Details}
       & \textbf{if } & \rel \text{ is \emph{binary}}
\end{align*}

% \begin{figure}[h]
%   \centering
%   \fbox{ \input{../code/AUCSP/src/Document/tikz/ContextAssess} }
%   \caption{Binary relations within an information graph. One can
%            distinguish the relations between the assessed information pieces
%            and the relations between assessed and the known ones.
%           }
% \end{figure}

\medskip
\noindent
Contexts may have restrictions, that would compromise a potential solution
because of lack of information during assessment. A good example is
\emph{minimum classes} relation for \emph{professors}, that would force
corresponding agents to immediately reject propositions, that don't have enough
professor's classes, without considering a possibility of joining them to achieve
full constraints compliance.

Coherence \textbf{assessment \emph{modes}} are introduced to counter the problem.
All coherence assessment functions should provide \emph{mode} argument, that
\begin{enumerate}
  \item Is \underline{passed to each context}, when filtering candidates.
  \item Is \underline{passed to each relation}, when assessing a candidate at a context.
  \item Can \underline{change relation evaluation}, weakening or disabling the
        restriction described.
\end{enumerate}

\bigskip

\noindent
The contexts stand for different constraints and can be divided into
\begin{itemize}[leftmargin=2cm]
  \item[Common:] same for all agents, represent \emph{class coherence}:
    \begin{itemize}
      \item \emph{capabilities}: class structure ($\restrC$);
      \item \emph{time consistency}: relations between classes ($\restrT$);
    \end{itemize}
  \item[Internal:] personal configuration, unique for each agent:
    \begin{itemize}
      \item \emph{obligations}: strong restrictions ($\restrW$);
      \item \emph{preferences}: weak restrictions ($\restrS$);
    \end{itemize}
  \item[External:] communicates with the ``neighboring'' agents and fetches
    their opinions about the candidates, thus cooperating towards the
    \emph{common goal}. It doesn't depend on agent's role or
      any agent-specific knowledge. It unites the constraints of various agents.
\end{itemize}


\subsection{Class Coherence}
\subfile{\rootdir/Solution/Contexts/ClassCoherence}
\subsection{Internal Contexts}
\subfile{\rootdir/Solution/Contexts/InternalCoherence}
\subsection{External}
\subfile{\rootdir/Solution/Contexts/ExternalCoherence}


\end{document}
