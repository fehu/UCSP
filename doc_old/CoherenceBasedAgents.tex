\section{Coherence-based Agents}

\red{ They are not the same as ``Coherence-driven Agents'' from \cite{UAB-Thesis}!
      It should be properly described.
 }

A \emph{coherence-based agent} makes its decisions, taking in account
the existing \emph{candidates} and their \emph{coherence}.
It must provide the contexts (with an order) and a \emph{decider}.
% It must contain it's contexts \red{?} within the state,
% define contexts order and set the splitting one.
% It also must provide a \emph{decider}.
\medskip

\noindent
Such agent acts repeatedly:
\begin{enumerate}
\item Gets its \emph{classes pool}, that may be stored within some context or
  somewhere else in the state.
\item Generates the initial candidates, using the \emph{splitting context} and
  classes pool.
\item Propagates the candidates through the contexts, as described in section
  \ref{section:coherence}.
\item Passes the \emph{assessed candidates} to the \emph{decider}.
\end{enumerate}

The decider must choose between two actions:
\begin{itemize}
\item Add new class(es) to the \emph{pool}. Such decision is taken if
  none of assessed candidates is \emph{acceptable}.
\item Select best solution and wait. This should be done if a satisfying
  solution was found. The agent's \emph{run state} is set to ``Paused'' and
  the \emph{status} is updated to ``Waiting''.
\end{itemize}


\subsection{Classes generation}
\label{section:class-gen}
When no \emph{acceptable candidates} can be obtained by combining the classes
from the \emph{pool}, new class need to be added (to the \emph{classes pool}).

In order to avoid ``garbage'' classes generation, the newly created classes must:
\begin{enumerate}[(1)]
  \item Be \emph{capacity} consistent. The group, professor and classroom,
    assigned to the class, must be \emph{capable} of handling the assigned discipline
    (it must comply with their respective \emph{capabilities}).
  \item Not be a repetition of any of \emph{discarted classes}
    (see section \ref{section:clean-pool}).
\end{enumerate}

In order to keep track of discarted classes without keeping them around,
class \emph{hashes} may be used. The only restriction on hash function is
injectiveness:
\begin{align*}
  \forall c_1, c_2 \in \text{ \emph{classes}} & \\
  \mathrm{hash}(c_1) &= \mathrm{hash}(c_2) \implies c_1 = c_2
\end{align*}

\subsection{Pool cleanup}
\label{section:clean-pool}

Even with no discipline-inconsistent classes being genrated
(sec. \ref{section:class-gen} (1)), the amount of
``bad'' classes (that worsen solutions) would be growing.
Those classes need to be discarted and their \emph{hashes} guarded
to be used in \ref{section:class-gen} (2). Discarted classes hashs must be
shared between \emph{group}, \emph{professor} and \emph{classroom} agents,
corresponding to the class.

A class should be considered \emph{solution worsening} and be discarted, if
\begin{enumerate}
  \item no \emph{acceptable candidate} (after propagating through all the contexts)
    contains the class;
  \item \red{?}
\end{enumerate}


\red{TO DO}
