\documentclass{beamer}

\providecommand{\rootdir}{../doc}

% \def\privateData{private_data}

% % % % % % % % % % % % % % % % % % % % % % % % % % % % % % % % % % % % % % % %

\usepackage[english]{babel}
\usepackage{ifthen}
\usepackage{color}
\usepackage{hhline}
\usepackage{adjustbox}
% \usepackage[utf8]{inputenc}
% \usepackage[backend=biber]{biblatex}

\usepackage{tikz} % , pgfplots, skak
\usetikzlibrary{fit, matrix, positioning, shapes, decorations.pathreplacing,
                shapes.geometric, chains, arrows, calc }




\newcommand{\red}[1]{{\color{red} #1}}


\newcommand{\resizeinput}[2][1]{%
  \resizebox{#1\textwidth}{!}{\input{#2}}%
}

% % % % % % % % % % % % % % % % % % % % % % % % % % % % % % % % % % % % % % % %

% https://tex.stackexchange.com/questions/45938/error-when-using-colour-in-author
\ifthenelse{\isundefined{\privateData}}
           {\author{\texorpdfstring{\color{red} Anonymous}{???}}}
           {\input{\privateData}}

% % % % % % % % % % % % % % % % % % % % % % % % % % % % % % % % % % % % % % % %

\title{University class schedule generation using personalizable agents negotiation}
\subtitle{Thesis for Master of Science in Intelligent Systems}
\institute[ITESM]{Tecnol\'{o}gico de Monterrey}
\logo{\includegraphics{../doc/escudo-itesm}}
\date{May, 2017}

% % % % % % % % % % % % % % % % % % % % % % % % % % % % % % % % % % % % % % % %

% \bibliography{\rootdir/References}

\mode<presentation>

\begin{document}

\frame{\titlepage}
\frame{\tableofcontents} % \frametitle{Outline}

% % % % % % % % % % % % % % % % % % % % % % % % % % % % % % % % % % % % % % I
\section{University Class Scheduling Problem (UCSP)}
\subsection{The Problem}

\begin{frame}{University Class Scheduling Problem}
  \begin{block}{The Problem}
    \textbf{U}niversity \textbf{C}lass \textbf{S}cheduling \textbf{P}roblem (UCSP)
    consists in finding valid \alert{class} assignations for \underline{all} the
    participants:
    \begin{itemize}
      \item Groups / Students
      \item Professors
    \end{itemize}
  \end{block}
  \begin{block}{Class}
    A class is an educational event, that is formed with purpose of studying
    some \alert{discipline}.
    \begin{columns}
      \begin{column}{4cm}
        \\Takes place
        \begin{itemize}
          \item in a \underline{classroom}
          \item on given \underline{day}
          \item during given \underline{time}
        \end{itemize}
      \end{column}
      \begin{column}{3cm}
        Links together
        \begin{itemize}
          \item a \underline{group} and
          \item a \underline{professor}
        \end{itemize}
      \end{column}
    \end{columns}

  \end{block}
\end{frame}

\begin{frame}
  \begin{columns}
    \begin{column}{4cm}
      \begin{block}{Schedule}
        The complete schedule consists of all the classes of all the participants,
        that can be seen as points in 5-dimensional space. It can be decomposed
        into a set of \alert{timetables}.
      \end{block}
    \end{column}
    \begin{column}{7cm}
        \resizeinput{\rootdir/img/ScheduleHypercube/GRPT-content.tikz}
    \end{column}
  \end{columns}
\end{frame}

\begin{frame}
  \begin{block}{Timetable}
    Timetable is projection of the schedule on the person/entity.
    It is a 2-dimensional table, that contains \underline{only} the classes
    of projection target.
  \end{block}
  \begin{columns}
    \begin{column}{.5\textwidth}
      \centering
      \begin{tabular}{|c||c|c|c|}
  \hline & Mon & Tue & $\cdots$ \\
  \hhline{|=#=|=|=|}
  08:30 -- 08:40 & x & & \\\hline
  08:40 -- 08:50 & x & & \\\hline
  08:50 -- 09:00 & x & & \\\hline
  $\vdots$\quad~--~\quad$\vdots$ & & & \\\hline
  09:50 -- 10:00 & x & & \\\hline
  10:00 -- 10:10 &   & & \\\hline
  10:10 -- 10:20 & y & & \\\hline
  10:20 -- 10:30 & y & & \\\hline
  $\vdots$\quad~--~\quad$\vdots$ & & & \\\hline
\end{tabular}

    \end{column}
    \begin{column}{.5\textwidth}
      \centering
      \begin{tabular}{|c||c|c|c|}
  \hline & Mon & Tue & $\cdots$ \\
  \hhline{|=#=|=|=|}
  08:30 -- 09:15 & x & & \\\hline
  09:25 -- 10:10 & x & & \\\hline
  10:30 -- 11:15 & y & & \\\hline
  11:25 -- 12:10 & y & & \\\hline
  Lunch          &   & & \\\hline
  13:25 -- 14:10 & z & & \\\hline
  14:20 -- 15:05 & z & & \\\hline
  15:25 -- 16:10 &   & & \\\hline
  $\vdots$\quad~--~\quad$\vdots$ & & & \\\hline
\end{tabular}

    \end{column}
  \end{columns}
  % \centering
  % \begin{tabular}{|c||c|c|c|c|c|c|}
  \hline & Mon & Tue & Wed & Thu & Fri & Sat \\
  \hhline{|=#=|=|=|=|=|=|}
  08:00 -- 08:30 & x & & & w & & \\\hline
  08:30 -- 09:00 & x & & & w & & \\\hline
  09:00 -- 09:30 & x & & & z & & \\\hline
  09:30 -- 10:00 & y & & & z & & \\\hline
  10:00 -- 10:30 & y & & & z & & \\\hline
  $\vdots$\quad~--~\quad$\vdots$ & & & & & & \\\hline
  21:00 -- 21:30 &   & & & & & \\\hline
  21:30 -- 22:00 &   & & & & & \\\hline
\end{tabular}

\end{frame}

% % % % % % % % % % % % % % % % % % % %
\subsection{Constraints}

\begin{frame}{University Class Scheduling Constraints}
  \fbox{
    \begin{columns}[t]
      ~~Independent
      \begin{column}{.4\textwidth}
        \begin{block}{Class}
          \begin{itemize}
            \item Group need
            \item Professor can teach
            \item Classroom is suitable
          \end{itemize}
        \end{block}
      \end{column}
      \begin{column}{.4\textwidth}
        \begin{block}{Time}
          \begin{itemize}
            \item Classes non intersection
          \end{itemize}
        \end{block}
      \end{column}
    \end{columns}
  }
  \\[.7cm]
  \fbox{
    \begin{columns}[t]
      Personal
      \begin{column}{.4\textwidth}
        \begin{block}{Obligations}
          \begin{itemize}
            \item Personal \alert{strong} restrictions
          \end{itemize}
        \end{block}
      \end{column}
      \begin{column}{.4\textwidth}
        \begin{block}{Preferences}
          \begin{itemize}
            \item Personal \alert{weak} restrictions
          \end{itemize}
        \end{block}
      \end{column}
    \end{columns}
  }
\end{frame}

% % % % % % % % % % % % % % % % % % % % % % % % % % % % % % % % % % % % % % II
\section{Constraint Satisfaction Problem (CSP)}
% \subsection{CSP Definition}
% \subsection{CSP Examples}
% \subsection{CSP Solution Methods}

\begin{frame}{Constraint Satisfaction Problem}
  \begin{block}{Definition}

  \end{block}
  \begin{examples}
    ???
  \end{examples}
\end{frame}

\begin{frame}{CSP Solution Methods}
  ???
\end{frame}

% % % % % % % % % % % % % % % % % % % % % % % % % % % % % % % % % % % % % % III
\section{Formal Definition}
% \subsection{Define Entities and Constraints}
% \subsection{Define Classes and Candidate}
% \subsection{Partial and Complete Solutions}

\begin{frame}{University Class Scheduling Problem}
  \begin{columns}[t]
    \begin{column}{.4\textwidth}
      \begin{block}{Disciplines $\{d_i\}$}
        \begin{itemize}
          \item academic program
          \item special requirements
        \end{itemize}
      \end{block}
    \end{column}
    \begin{column}{.4\textwidth}
      \begin{block}{Groups $\{g_i\}$}
        \begin{itemize}
          \item disciplines needed
        \end{itemize}
      \end{block}
    \end{column}
  \end{columns}
  \begin{columns}[t]
    \begin{column}{.4\textwidth}
      \begin{block}{Classrooms $\{r_i\}$}
        \begin{itemize}
          \item capacity
          \item equipment
        \end{itemize}
      \end{block}
    \end{column}
    \begin{column}{.4\textwidth}
      \begin{block}{Professors $\{p_i\}$}
        \begin{itemize}
          \item able to teach (disciplines)
        \end{itemize}
      \end{block}
    \end{column}
  \end{columns}
  \begin{columns}
    \begin{column}{.4\textwidth}
      \begin{block}{Working Days $\{\bar{d}_i\}$}\end{block}
    \end{column}
    \begin{column}{.4\textwidth}
      \begin{block}{Discrete Time $\{\bar{t}_i\}$}\end{block}
    \end{column}
  \end{columns}
\end{frame}

\begin{frame}{University Class Scheduling Problem}
  \begin{block}{Class Core $\mathrm{cc} \sim \left< d, g, p \right>$}
    \emph{possible} class (of discipline $d$) for group $g$,
    taught by professor $p$
  \end{block}
  \begin{block}{Class $ c \sim \left< \mathrm{cc}, r, \bar d, \bar t_b, \bar t_e \right> $}
    \emph{concrete} class, with day, time (beginning \& end) and room
  \end{block}
  \begin{block}{Candidate $\tilde{c}=\{c\}$}
    set of \alert{valid} classes, assessed using \underline{coherence}
    %{Candidate $\tilde{c}_a=\{c\}$}
    % set of \alert{valid} classes for agent $a$
  \end{block}
\end{frame}

\frame{
  \trimbox{30pt 0 0 0}{
    \resizeinput{\rootdir/img/ConnectionMatrix/Candidates-content.tikz}
  }
}

% % % % % % % % % % % % % % % % % % % % % % % % % % % % % % % % % % % % % % IV
\section{Coherence}
% \subsection{Coherence Usage in Thesis}
% \subsection{What is Coherence}
% \subsection{Thagard, usage examples}
% \subsection{Contexts Introduction}
% \subsection{Contexts Usage}
% \subsubsection{Independent}
% \subsubsection{Internal}
% \subsubsection{External}

\begin{frame}
  \begin{block}{Coherence}
    In simple worlds, \emph{coherence} is a state or measure of some \emph{pieces}
    fitting together into a whole. Coherence can be applied to almost any aspect of
    the universe.
  \end{block}
  \begin{quote}
    The consistency or \emph{coherence} within a logical or mathematical
    system, means that $p$ and $\neg p$ must not be derivable from the
    basic assumptions in accordance with the observance of the syntactical
    rules \cite{Daya60}.
  \end{quote}
  \begin{quote}
    Coherence theory is a psychologically motivated motivational cognitive theory with
    foundations in philosophy that approaches problems in terms of the satisfaction
    of multiple constraints within networks of highly interconnected elements
    \cite[p.~19]{UAB-Thesis}.
  \end{quote}
\end{frame}

% % % % % % % % % % % % % % % % % % % %
\subsection{Contexts}


% % % % % % % % % % % % % % % % % % % % % % % % % % % % % % % % % % % % % % V
\section{Agents}
\subsection{Agent Definition}
\subsection{Solve CSPs with Agents}
\subsection{Distribution, Communication and Roles}

% % % % % % % % % % % % % % % % % % % % % % % % % % % % % % % % % % % % % % VI
\section{Agent Roles}

\subsection{Group}
\subsubsection{Candidates Creation}
\subsubsection{Propagation through Contexts}
\subsubsection{Placement Intention and Conflicts}
\subsubsection{Candidate Modification}
\subsubsection{Sleep and Awake, Solution Improvement}

\subsection{Professor}
\subsubsection{Reactive}
\subsubsection{Solution Improvement}

\subsection{Schedule Holders}
\subsubsection{Putting Candidates Together}
\subsubsection{Conflicts and Forced Resolution}
\subsubsection{Schedule Observation}

\subsection{Common / Conflicts Resolution}
\subsubsection{Deep Coherence Assessment}
\subsubsection{Candidate ``Rareness''}

% % % % % % % % % % % % % % % % % % % % % % % % % % % % % % % % % % % % % % VII
\section{Improvements}
\subsection{Students Representation}
\subsection{Classrooms Representation}
\subsection{Roles Extension}

% % % % % % % % % % % % % % % % % % % % % % % % % % % % % % % % % % % % % % VIII
\section{Results}
%% ??? TODO ???



\begin{frame}{References}

\end{frame}

\end{document}
