%\begin{document}


% \def\lifetime    {\mathit{lifetime}}
% \def\class       {\mathit{class}}
% \def\execute     {\mathit{execute}}
% \def\decide      {\mathit{decide}}
% \def\knownClasses{\overset{\mathrm{known}}{\mathit{classes}}}


\section{Agents and Negotiation}

The \emph{negotiating agents} are isolated proactive computational entities,
capable of sending and receiving messages.

The \emph{isolation} denotes that agents' internal states are protected
from outside access. Messaging is the only way an agent can be interacted with.
To ensure it, an agent is normally never interacted with directly, but
communicated via \emph{agent reference} instead.
The \emph{pro-activity} implies a capacity of acting asynchronously,
with no ``external'' cause.

Agents have \emph{roles}, that describe whom or what an agents represents
in the negotiation.

A negotiation has a \emph{classes pool}, distributed between the agents.
A class $\left< d, g, p, r, \bar d, t, t \right>$ is known to agents,
representing $g$, $p$ and $r$.

Negotiations are organized by special kind of agents --- \emph{controllers}.


\subsection{Agents}

In it's state an agent holds the \emph{run state} and special \emph{shared
  state}, called \emph{status}. The run state controls agent's execution
(and termination), while the status is monitored by the respective controller.
Agents have three \emph{run states}: ``Paused'', ``Running'' and ``Terminate''.
The first two states control pro-action; setting the last one causes all
agent's processes to stop. The \emph{status}es are further described in
section \ref{agentControllers}.

\subsubsection{Behavior}

Agent's behavior is flexible and may be changed during agent's lifetime.
It is determined by \emph{message handling} and \emph{action} functions.

An agent can simply \emph{send} a message, or it can \emph{ask} for some
\emph{expected response}, corresponding to the message sent. Agents have
separate handler functions for the two cases. All behavior functions
accept two arguments:
\begin{enumerate}
\item the agent's \emph{inner interface}, that provides
  control over agent's own behavior and lifetime;
\item agent's \emph{internal state}.
\end{enumerate}

Agents are executed in two computational threads: one for message handling
and another for proactive action.

\subsubsection{Coherence-based}

A \emph{coherence-based agent} bases its decisions on existing \emph{candidates}
and their \emph{coherence}. It must contain it's contexts within the state,
define contexts order and set the splitting one. It also must provide a
\emph{decider}.
\medskip

\noindent
Such agent acts repeatedly:
\begin{enumerate}
\item Gets its \emph{classes pool}, that may be stored within some context or
  somewhere else in the state.
\item Generates the initial candidates, using the \emph{splitting context} and
  classes pool.
\item Propagates the candidates through the contexts, as described in section
  \ref{section:coherence}.
\item Passes the \emph{assessed candidates} to the \emph{decider}.
\end{enumerate}

The decider must choose between two actions:
\begin{itemize}
\item Add new class(es) to the \emph{pool}. Such decision is taken if
  none of assessed candidates is \emph{acceptable}.
\item Select best solution and wait. This should be done if a satisfying
  solution was found. The agent's \emph{run state} is set to ``Paused'' and
  the \emph{status} is updated to ``Waiting''.
\end{itemize}

\subsubsection{Classes generation}
\label{section:class-gen}
When no \emph{acceptable candidates} can be obtained by combining the classes
from the \emph{pool}, new class need to be added (to the \emph{classes pool}).

In order to avoid ``garbage'' classes generation, the newly created classes must:
\begin{enumerate}[(1)]
  \item Be \emph{capacity} consistent. The group, professor and classroom,
    assigned to the class, must be \emph{capable} of handling the assigned discipline
    (it must comply with their respective \emph{capabilities}).
  \item Not be a repetition of any of \emph{discarted classes}
    (see section \ref{section:clean-pool}).
\end{enumerate}

In order to keep track of discarted classes without keeping them around,
class \emph{hashes} may be used. The only restriction on hash function is
injectiveness:
\begin{align*}
  \forall c_1, c_2 \in \text{ \emph{classes}} & \\
  \mathrm{hash}(c_1) &= \mathrm{hash}(c_2) \implies c_1 = c_2
\end{align*}

\subsubsection{Pool cleanup}
\label{section:clean-pool}

Even with no discipline-inconsistent classes being genrated
(sec. \ref{section:class-gen} (1)), the amount of
``bad'' classes (that worsen solutions) would be growing.
Those classes need to be discarted and their \emph{hashes} guarded
to be used in \ref{section:class-gen} (2). Discarted classes hashs must be
shared between \emph{group}, \emph{professor} and \emph{classroom} agents,
corresponding to the class.

A class should be considered \emph{solution worsening} and be discarted, if
\begin{enumerate}
  \item no \emph{acceptable candidate} (after propagating through all the contexts)
    contains the class;
  \item \red{?}
\end{enumerate}


\red{TO DO}





\subsection{Controllers}

The controllers create new negotiating agents, and monitor their \emph{status}es
over their lifetimes. Controllers also provide communication references for the
agents they control. In order to facilitate references sharing, the controllers
are restricted to a single \emph{role} for it's agents.

Controllers have hierarchical structure, starting with the \emph{root
  controller}. The orders received by \emph{root} are propagated to
the underlying controllers.

\medskip

As mentioned before, negotiating agents have a special state,
called \emph{status}, that is monitored by the controllers. Apart
from receiving agents' errors, the \emph{status}es are used to
extract the \emph{distributed solution} from the agents.
This is done using ``Waiting'' and ``Locked'' statuses.

The agents set ``Waiting'' status when a suitable solution is found,
and the corresponding controller waits for all of its subjugated
agents to set it. When it is done, the controller sends ``Lock''
command to the agents and notifies the \emph{parent controller}.
When the \emph{root controller} receives ``Lock'' notifications from
all the controllers, it sends ``Collect'' order. On that order, the controllers
extract the solution parts (guarded in the statuses) and propagate them up to
the root.







% An agent is executing repetitively it's \emph{lifetime} function, while it remains
% capable of responding messages.

% \begin{align*}
%   \knownClasses &: \{\class_i\} \\
%   \lifetime &:~ \knownClasses \mapsto * \\
%   \lifetime &=  \execute \circ \decide \\
%   \decide   &:~ \knownClasses \mapsto
%               \mathrm{New\,candidate} ~|~
%               \mathrm{Ask\,acceptance} ~|~
%               \mathrm{Ask\,to\,yield}
% \end{align*}

% \subsection{New candidate}
% When no satisfying \emph{candidate} was found, new \emph{candidate}
% needs to be aggregated to the \emph{known classes}.
% The candidate must be \emph{internally coherent}.

% The newly created \emph{classes} (of which consists a candidate) are
% sent to the corresponding professor, group and classroom (except self).
% A confirmation, based on \emph{capabilities} must be received from all
% the candidates. Otherwise, another candidate must be generated.

% \textit{A cache might be used for the capabilities.}


% \subsection{Ask acceptance}
% When a candidate has no no contradiction at any context, an agent may decide to
% select it as it's part of solution and wait for the rest to do the same.
% It can be woken by a successful yield demand, forcing the waiting agent to
% search for a new solution.


% \subsection{Ask to yield}
% It is a solution when small percent of counterparts responded with a bad
% opinion (\emph{external} context) about a candidate. It asks them to yield
% to the candidate, that would mean (??) deleting (??) the contradicting classes
% in case of acceptance.  All asked agents must respond positively in order for
% the deletion to take place. In case of any refusal, the initial agent must
% (??) delete (??) the contradicting classes instead.

% The decision whether to yield or not is made considering the best inner
% coherence of any configuration, containing any contradicting class.

%\end{document}
