\documentclass[ThesisDoc]{subfiles}

\providecommand{\rootdir}{.}

\def\domain{\mathrm{domain}}

\def\domain{\mathrm{domain}}
\def\pop{\mathrm{pop}}

\def\behaviour{\mathrm{behaviour}}
\def\act{\mathrm{act}}
\def\react{\mathrm{react}}
\def\state{\mathrm{state}}
\def\action{\mathrm{action}}
\def\msg{\mathrm{message}}

\def\coh{\mathrm{coh}}
\def\cohi{\mathrm{\widetilde{coh}}}
\def\rel{\mathrm{rel}}
\def\fold{\mathit{fold}\,}

\def\restrC{\accentset{C}{\xi}}
\def\restrT{\accentset{T}{\xi}}
\def\restrS{\accentset{S}{\xi}^p_i}
\def\restrW{\accentset{W}{\xi}^p_i}

\def\ctx{\mathit{ctx}}
\def\codom{\mathrm{codomain}}
\def\maybe{\mathrm{Maybe\,}}


\begin{document}

% \green{Explicar que son los CSP}
\chapter{Constraint Satisfaction Problems}
\label{chapter:csp}

% \green{Poner una definición no formal}
% \medskip

There are many problems that require positioning or assigning something,
respecting established \emph{restrictions}. \emph{Graph coloring} and
\emph{n-queen} chess problem are classical constraint satisfaction problems (CSPs).

% \medskip
% \green{Poner algunos ejemplos}
% \medskip

The graph coloring problem comes from cartography, where it was needed to color
countries on political maps, in such a way that no country had a land border with
a country of the same color. It was found that \textbf{any} map can be colored
with only four colors.

The n-queen problem is known in chess as \emph{eight queen puzzle}.
Queens in chess can move/attack to/at any square,
that is in a the same row, column or diagonal with the queen.
In the puzzle one needs to place eight (the size of a chess desk)
queens on the desk, so that none of them is threatened by another.
N-queens problems is a generalization of that puzzle, where $n$ queens need
to be placed on a $n \times n$ desk.

Constraint satisfaction problems are found in many areas:
machine vision, natural language processing, theorem proving,
planning and in our problem --- scheduling \cite{MAS}.


\section{Formal definition}
%%%%%%%%%%%%%%%%%%%%%%%%%%%%%%%%%%%%%%%%%%%%%%%%%%%%%%%%%%%%%%%%%%%%%%%%%%%%%%%%
\noindent
Formally speaking, a CSP is defined by its \emph{variables} $V$ with the
corresponding \emph{domains} and the \emph{constraints} $\{\xi\}$
over values assignation \cite{MAS}.

\begin{align*}
  V                &= \{v_i\}_{i=1}^N
& \{{\dot v}^i_j\} &= \domain(v_i)
& \xi              &: \{{\dot v}^i_\ast\}_{i=1}^N \mapsto [0,1]
\end{align*}


A variable defines a ``slot'' that can hold a value from the corresponding domain.
A solution to CSP is an assignation of the values ${\dot v}^i_\ast$
to the variables $V$, such that all the restriction hold.
\begin{equation}
  {\dot V} = \{{\dot v}^i_\ast\}_{i=1}^N \text{is a solution}
   \iff \forall \xi \in \{\xi\} \Rightarrow \xi({\dot V}) = 1
\end{equation}

The constraints above are defined in the most generalized form,
over the entire solution. There is a particular case, that is found in many problems
--- \emph{binary constraints}, imposed on \emph{pairs} of values:
$$\xi_2 : \left< {\dot v}^i_\ast, {\dot v}^j_\ast \right> \mapsto \{0,1\}$$

If all constrains are binary, than the satisfaction condition is
$$\forall \xi_2     \in \{\xi\},~
  \forall v_i, v_j  \in V | v_i \not= v_j
~ \Rightarrow ~ \xi_2(v_i, v_j) = 1
$$

\medskip

In the graph coloring problem, the variables are the \emph{colors to be assigned}
for each graph node $\{n_i\}_{i=1}^N$. In this case, all the variables have
the same domain values: four colors, for example \textit{Red}, \textit{Green},
\textit{Blue} and \textit{Yellow}. The constraint is binary and depends on graph
structure:
$$ \xi_2(x,y)= \begin{cases}
  0 & \mbox{if } \exists \text{~edge~} x \leftrightarrow y
                ~\land \text{~color~} x = \text{color~} y \\
  1 & \text{otherwise}
\end{cases}
$$

% For the n-queens problem, there are two ways of representing the variables.One
% can see queens' positions as a single variable --- pairs $\left< x,y \right>$.
% But it's better to handle them separately, as it would be shown in \ref{TODO}.
% In this case each queen would have two variables:
% horizontal and vertical positions $x$ and $y$.

For the n-queens problem, the variables are queens' positions ---
pairs $\left< x,y \right>$, where $x$ is queen's horizontal position and
$y$ is the vertical one. The restrictions can be gathered within a single
constraint function:
$$\xi_2(\left<x_1,y_1\right>, \left<x_2,y_2\right>) =
    \begin{cases}
      0 & \mbox{if } \begin{cases}
                        &      x_1 = x_2  \lor y_1 = y_2 \\
                        \lor~& x_1 = y_2 \land x_2 = y_1 \\
                        \lor~& x_1-y_1 = x_2-y_2\\
                        \lor~& x_1+y_1 = x_2+y_2
                     \end{cases} \\
     1 & \text{otherwise}
    \end{cases}
$$

\begin{figure}
  \begin{subfigure}[b]{0.3\textwidth}
    \centering
  	\subfile{\rootdir/img/NQueens/Restrictions.tikz}
  	\caption{Queen moves/attacks.}
  	\label{fig:QueenMoves}
  \end{subfigure}
  \hfill
  \begin{subfigure}[b]{0.3\textwidth}
    \centering
  	\includegraphics[trim=100 260 370 370, clip] % width=\textwidth
               {\rootdir/img/easteuro}
  	\caption{Part of Central Europe \cite{UN-CEU-Map}.}
  	\label{}
  \end{subfigure}
  \hfill
  \begin{subfigure}[b]{0.3\textwidth}
  	\centering
    \subfile{\rootdir/img/ColorMap/CEUMapGraph.tikz}
  	\caption{Graph coloring problem.}
  	\label{fig:ColoringGraph}
  \end{subfigure}
  \caption{CSP examples}
\end{figure}

  In our \emph{university classes scheduling} problem (UCSP) the variables
are personal schedules, called \emph{timetables} in this thesis,
for each \emph{participant}: professor and group/student.
  A timetable consists of the day-time slots, where classes can be put.
  A \emph{class} is formed for teaching a group a specific \emph{subject},
connecting the participants (of each kind) and establishing a classroom and
the beginning--end \emph{time}.
  Problem \emph{constraints} can be divided into:
\begin{enumerate}
  \item \underline{Class constraints}: a class should be a productive event, so
    a professor should be able to \emph{teach} the class, the classroom should
    have the \emph{capacity} to hold all the students and be properly \emph{equipped},
    and the group should be \emph{enrolled} to class subject
    (further called \emph{discipline}).
  \item \underline{Time constraints}: no participant can have two classes
    at the same time (or intersecting in time), as mentioned before.
  \item \underline{Strong restrictions}: the restrictions, put on a participant, that
    \emph{must} be respected. May include working hours, fixed lunch recess time,
    and any other institution or person specific \emph{obligations}.
  % \item \red{\underline{Weak restrictions} (MOVE)}: the restrictions, that \emph{should} be respected,
  %   but are not critical to the solution. Compliance with these restrictions
  %   raises \emph{solution quality}, but it is assumed that the all of them
  %   cannot be fully met for all the participants --- they are intended to
  %   represent personal \emph{preferences}.
\end{enumerate}

\bigskip

%%%%%%%%%%%%%%%%%%%%%%%%%%%%%%%%%%%%%%%%%%%%%%%%%%%%%%%%%%%%%%%%%%%%%%%%%%%%%%%%
  Until now we were speaking about restrictions satisfaction, but the problem can
be extended to an \emph{optimization} one by defining an
\emph{objective function} over the values configurations.

  Unlike the restrictions (that yield boolean result), the objective functions
should have continuous codomains:
$$\tilde\xi : \{{\dot v}^i_\ast\}_{i=1}^N \mapsto \Re$$
  In order to facilitate the calculations and erase the internal differences of
the objectives, it's often required that the functions must be normalized:
the co-domains are restricted to $[-1,1]$ or $[0,1]$ intervals.

  For example, in case of our UCSP,
one can add \underline{personal preferences} or some institution criteria as
optimization parameters.
  Compliance with the preferences raises \emph{solution quality},
but it is assumed that the all of them cannot be fully met for all the
participants, because \emph{personal} preferences are often
contradictory for different persons.

\section{Solution Methods}
%%%%%%%%%%%%%%%%%%%%%%%%%%%%%%%%%%%%%%%%%%%%%%%%%%%%%%%%%%%%%%%%%%%%%%%%%%%%%%%%
% \green{Explicar por qué son importantes computacionalmente}\\
% \medskip

% \noindent
  Solving these problems usually presents difficulties due to the amount of
possible combinations to consider for a solution. Let's have a look at
some university schedule.
  For example, six working days in a week and twelve time slots every day
(every hour from 8:00 to 20:00), would yield $6 \times 12 = 72$ options
to place each class.
  Given that each group needs, for example, five different disciplines to be assigned,
there would be $\multibinom{72}{5} = \num[group-separator={,}]{18474840}$
possible classes assignments for each group,
even without considering professors and classrooms.

% \bigskip
% %%%%%%%%%%%%%%%%%%%%%%%%%%%%%%%%%%%%%%%%%%%%%%%%%%%%%%%%%%%%%%%%%%%%%%%%%%%%%%%%
% \green{Explicar formas de resolver los CSP} \todo
\medskip
\noindent

  The researchers have been looking for means of solving such
problems within reasonable time.
  During the last years different algorithms and techniques where developed,
such as
\emph{dynamic constraint satisfaction based on extension particle swarm
      optimization algorithm} \cite{CSPswarm},
\emph{dynamic state bounding} \cite{CSPdynStateBound},
\emph{conflict-vector detection} \cite{CSPtimetable},
\emph{neural networks} \cite{CSPneuro},
\emph{ant colony optimization} \cite{CSPcunningACO, CSPlimmemACO},
\emph{selective hyper-heuristics} \cite{CSPhypHeur}
and \emph{agents} \cite{CSPagent2013, CSPagent2014, DCSPagent1998}.

\medskip

Those methods do not seek for the optimal solutions, focusing instead on
``rather good'' ones, that can be obtained within reasonable time using admissible
computing resources.

\bigskip

\noindent
Some agent solution methods are presented in section \ref{sec:CSP-Agents}.

% \todo \red{: write something about one of the techniques}



\section{Distributed problems}
%%%%%%%%%%%%%%%%%%%%%%%%%%%%%%%%%%%%%%%%%%%%%%%%%%%%%%%%%%%%%%%%%%%%%%%%%%%%%%%%
Agents approach allows solving CSPs in a distributed manner by distributing
the problem knowledge and constraints
\cite{DCSPagent1998, DCSP2013, CSPagent2014, MAS, MAS-Survey}.

\begin{displayquote}[\cite{MAS-Survey}] % \cite[p.~2]{MAS-Survey}
  For example, a constraint satisfaction problem can often be
  decomposed into several not entirely independent
  subproblems that can be solved on different processors. $\dots$
  MAS (Multiagent System) allows the sub-problems of a constraint satisfaction
  problem to be subcontracted to different problem solving agents with their own
  interests and goals.
\end{displayquote}

\noindent
Such problems are \emph{Distributed Constraints Satisfaction Problems} (DCSPs).
\begin{displayquote}[\cite{MAS}] % \cite[p.~4]{MAS}
  In a distributed CSP, each variable is owned by a different agent. The goal is
  still to find a global variable assignment that meets the constraints, but each agent
  decides on the value of his own variable with relative autonomy. While he does
  not have a global view, each agent can communicate with his neighbors in the
  constraint graph.
\end{displayquote}


University class scheduling problem fits into a the DCSP class:
\begin{itemize}
  \item Some constraints are agent-specific: strong restrictions and
        personal preferences.
  \item Classes assignments in form of \emph{candidates} are distributed between
        the negotiating agents (representing groups and professors).
\end{itemize}

The original UCSP is decomposed in a set of sub-problems, represented by some
participants. A sub-problem consist in resolving constraints over an agent's
\emph{candidate}, while considering the \emph{opinions} of \emph{neighboring}
agents.

The sub-problems are then recomposed into a schedule, resolving any conflicts
between the sub-solutions with the established rules.

\end{document}
