\documentclass[../header]{subfiles}

\begin{document}

\providecommand{\rootdir}{..}

\def\domain{\mathrm{domain}}

\def\domain{\mathrm{domain}}
\def\pop{\mathrm{pop}}

\def\behaviour{\mathrm{behaviour}}
\def\act{\mathrm{act}}
\def\react{\mathrm{react}}
\def\state{\mathrm{state}}
\def\action{\mathrm{action}}
\def\msg{\mathrm{message}}

\def\coh{\mathrm{coh}}
\def\cohi{\mathrm{\widetilde{coh}}}
\def\rel{\mathrm{rel}}
\def\fold{\mathit{fold}\,}

\def\restrC{\accentset{C}{\xi}}
\def\restrT{\accentset{T}{\xi}}
\def\restrS{\accentset{S}{\xi}^p_i}
\def\restrW{\accentset{W}{\xi}^p_i}

\def\ctx{\mathit{ctx}}
\def\codom{\mathrm{codomain}}
\def\maybe{\mathrm{Maybe\,}}



\subsection{Quantum-inspired Genetic Algorithm}
\label{sec:QGA}
An interesting modification of Genetic Algorithm (GA), based on the concept and
principles of \emph{quantum computing} --- QGA, was proposed in article
\cite{QuantumGeneticAlgorithm}.
Instead of being represented by a list of \emph{genes}, as in classical GAs,
in QGA a chromosome is represented by a \emph{Q-bit individual}.

\emph{Q-bit} is the quantum counterpart of the classic bit --- the smallest
unit of information. While a normal \emph{bit} can take either 0 or 1 values,
a \emph{Q-bit} can take any \emph{superposition} of the tho states, including
margin values: states 0 and 1 \cite{QuantumGeneticAlgorithm}.
The main difference is that such ``bit'' can be
0 and 1 at the same time, just like famous Schrodinger's cat, that is both dead
and alive. The state of a Q-bit can be represented by two complex numbers
$\alpha$ and $\beta$, that denote the probabilities of a Q-bit being found in
states 0 and 1 respectively.
The values must be normalized: $|\alpha|^2 + |\beta|^2 = 1$.

A \emph{Q-bit individual} of $m$ Q-bits is defined as
\begin{equation*}
\left[
  \begin{array}{cccc}
  \alpha_1 & \alpha_2 & \dots & \alpha_m \\
  \beta_1  & \beta_2  & \dots & \beta_m  \\
  \end{array}
\right]
\end{equation*}

One of the practical advantages, claimed my the article, is:
\begin{displayquote}
  Evolutionary computing with Q­-bit representation has a better characteristic
  of population diversity than other representations, since it can represent
  linear superposition of states probabilistically.
\end{displayquote}
It presents an example of a three-Q-bit individual, and shows its equivalence
to 8 classical chromosomes of 3 genes at once.

\medskip\noindent
Major differences are found in the \emph{genetic operators} applied:
neither crossover or mutations are used. Instead, tournament selection strategy
with elite retaining model is used to generate new populations.
It works as follows:
\begin{enumerate}
  \item Tournament set --- a set $K$ individuals, is randomly selected from
        the population.
  \item A random number $r$ (between 0 and 1) is selected.
  \item If $r < 0.8$ (empirically established by the authors), then the fittest
        individual in the tournament set is chosen for reproduction.
        Otherwise, any chromosome is chosen from the tournament set.
\end{enumerate}
\noindent
After selecting intermediate population of the same size, \emph{quantum rotation gate}
is applied to each individual's Q-bits, creating new population.
The angle used is selected from a table, provided by the authors.

While not providing explicitly any stop criteria for the proposed algorithm,
the authors have compared their method with classical GA, using same number
of generation as stop criterion in both.
University classes scheduling problem was selected for the tests.
The presented results show better final fitness and faster convergence rate for
the proposed QGA.

\end{document}
