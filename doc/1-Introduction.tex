\documentclass[header]{subfiles}
\begin{document}

\chapter{Introduction}

% \todo
%
% \medskip
%
% Class scheduling is a \emph{constraint satisfaction problem} (CSP) ---
% a large kind of problems, that is properly discussed in chapter \ref{chapter:csp}.
%
% \medskip
%
% \noindent
% A solution for general UCSP problem
% (formalized in chapter \ref{sec:ProblemFormal}) \red{was implemented}
% (see chapter \ref{chapter:solution}) \red{and tested} (see chapter \ref{chapter:test}).
% Possible improvements are discussed in chapter \ref{chapter:improvements}.

% % % % % % % % % % % % % % % % % % % % % % % % % % % % % % % % % % % % % % % %
\section{Motivation}
Creating coherent schedule for a university is a painstaking manual work, that
the administrative staff has to repeat every term, using its experience and
intuition. In general, class assignment problem has a great number of possible
solutions, that are only limited by schedule consistence. The simplest way to
represent complete university schedule, is using \emph{timetables} to represent
each person's schedule for a week. Such personal schedule might be found adequate
by a person, or not, regardless of schedule formal consistence.

Second motivation is research in \emph{concurrent} problem solving, that is
why \underline{agents negotiation} was chosen as solution method.

% % % % % % % % % % % % % % % % % % % % % % % % % % % % % % % % % % % % % % % %
\section{Problem Statement and Context}
Solving \textbf{U}niversity \textbf{C}lass \textbf{S}cheduling \textbf{P}roblem (UCSP)
consists not only in finding some valid schedule, but also in providing certain
grade of \emph{mean subjective fitness}\footnote{%
  \emph{subjective fitness} is schedule assessment, from the point of
  view of a person; different persons' opinions are combined into
  \emph{mean subjective fitness}.%
}, determined by living people.
The formal properties of a schedule \emph{validness} are
\begin{enumerate*}[1)]
  \item class internal consistency, that stands for class formation logic;
  \item class consistency, that ensures classes non-intersection;
  \item time restraints
    \begin{enumerate*}[(a)]
      \item over the whole schedule,
      \item for some specific participants.
    \end{enumerate*}
\end{enumerate*}
The informal, \emph{subjective}, opinions about schedules should be provided by
the interested parties in form of \emph{preference} functions. Such functions
should provide the attitude of their creators to the selected aspects of
\underline{personal} schedules. Attitude values should lie within unit interval
$(0,1]$, where $1$ stands for total approval.

While the essential part of all UCSPs is the same, actual classes assignment
policies, timetables used and scheduling participants might vary from institution
to institution. In this \thisdoc\ \underline{a general} problem is researched
first and then possible improvements are described.

\bigskip
Generic \underline{university class scheduling problem} consists in finding a
\emph{university schedule}, that is formed of all participants' (and some entities')
\emph{personal schedules}, by assigning \emph{classes} between \emph{student groups}
and \emph{professors} in certain \emph{classroom} at given \emph{day} and
\emph{time}. It is a \emph{Constraint Satisfaction Problem}.

\begin{itemize}
  \item Generic problem doesn't take into account the \emph{students}.
        Is is assumed that groups are preformed and \emph{fixed}, so that
        no student belongs to more than one group.
  \item All the professors are treated in the same manner, without concern for
        administrative details. Nevertheless, any participant can have
        \emph{time restrictions} established, that should be sufficient to
        represent, at least, the restrictions, caused by special cases,
        such as part-time job position.
\end{itemize}

\noindent
In chapter \ref{chapter:improvements} some extensions of the generic problem
are provided, including, most importantly, \emph{students representation}
(section \ref{sec:improvements-students}). This extension adds a preliminary
stage --- group formation. Groups become \emph{proxies}: coherence estimation,
performed by a group, would actually be done by its underlying students.

% % % % % % % % % % % % % % % % % % % % % % % % % % % % % % % % % % % % % % % %
\section{Solution Overview}
The proposed solution is based on \underline{agents negotiation}.
This method was chosen because agent paradigm is well suited for concurrent
execution. Each agent, as autonomous entity, can be executed in its own
isolated thread without any modification. The threads are connected by acts of
\emph{communication} (see chapter \ref{chapter:agents}), that is the base of
agents negotiation.

Internal agents behavior is based on the \emph{coherence} of the classes,
proposed by it and other agents. Coherence evaluation is divided in \emph{contexts},
that stand for different types of \emph{constraints}. The contexts can be divided
in \emph{internal} and \emph{external}. The former evaluate the underlying
constraints using only agent's own knowledge, while the latter, external context,
works towards the \emph{common goal}, by combining internal coherence of several
agents.
During internal coherence evaluation, an agent considers only the
classes that \emph{mention it} (that the agent is part of). In the same manner,
an agent proposes only the classes that it needs --- \emph{personal schedules}.
The schedules are then united and any arising conflict is resolved by comparing
\emph{total} schedule coherence, assuming the contesting proposal is accepted.
When all the agents agree on classes assignment (the classes for all the agents
are successfully united), the schedule is considered \emph{ready} and is reported.
After that the agents guard their respective internal coherence of the solution
to be used as a threshold, reset their proposals and begin to negotiate a better
solution, until explicitly stopped.


% % % % % % % % % % % % % % % % % % % % % % % % % % % % % % % % % % % % % % % %
\section{Further Document Structure}

\def\inChapter#1{In chapter \ref{chapter:#1}}
\def\showChapter#1{Chapter \ref{chapter:#1}}

\begin{itemize}[leftmargin=2.5cm]
  \item[\inChapter{csp}]
        Constraint Satisfaction Problems (CSP) are described and defined.
        Examples of such problems are given and various solution methods are
        listed. Some of the methods are described.
  \item[\inChapter{agents}]
        different definitions of agents are displayed. Negotiating agents are
        defined. Some CSP solving methods are presented, including an example.
  \item[\inChapter{Coherence}]
        the notion of coherence is discussed and some possible usages are given.
  \item[\showChapter{solution}]
        describes the proposed solution. Compact solution scheme can be found
        on figure \ref{fig:solution-flow}.
  \item[\inChapter{test}] test results are presented and discussed.
  \item[\showChapter{improvements}]
        describes what improvements can be made for the proposed solution and
        how.

\bigskip\noindent
The state of the art concerning the problem at hand can be found at sections
\ref{sec:CSP-solution-methods} and \ref{sec:MAS-UCSP}.

\end{itemize}

% \noindent
% This \thisdoc\ proposes an extendable UCSP solution with support for
% \emph{schedule personalization}. The latter is achieved by using agent
% negotiations between the representatives of each \emph{students/groups} and
% \emph{professors}. The representing agents permit defining
% \emph{personal preferences} and \emph{restrictions}, that are used to optimize
% the schedules.


\end{document}
