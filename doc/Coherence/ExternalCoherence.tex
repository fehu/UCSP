\subsection{External}

The external context asks counterparts an \emph{opinion} about a candidate.
An opinion is the \emph{inner coherence} of the agent being asked. This context
plays crucial role in coherence property \ref{eq:coh-fun-independ}. It combines
the internal coherences of the agent itself and other agents, mentioned in the
assessed candidate.

To speedup opinions assessment, the agents should share the newly created classes.
Any class $c_i \sim \left< \dots, g_i, p_i, r_i, \dots \right>$, created by
any agent of triple $\left< g_i, p_i, r_i \right>$, should be sent by that
agent to the rest of the triple. A received class must be added to agent's
classes pool.

\begin{figure}[h]
  \label{fig:CandidatesShareOpinions}
  \includegraphics[width=\textwidth]{img/CandidatesShareOpinions.png}
  \caption{Agents exchange thier \emph{internal coherences} of the candidates
            $\cohi^a_i = \cohi[a](c_i)$ in form \emph{opinions}.
          }
\end{figure}

The internal coherences are combined using \emph{common goal} function $\Gamma$.
The common goal must combine three coherence values (from a group, professor and
classroom), making no difference between value origins.
$$
  \Gamma(\coh_x^G, \coh_x^P, \coh_x^R) = \Gamma(\coh_x^G, \coh_x^R, \coh_x^P)
    = \dots = \Gamma(\coh_x^R, \coh_x^P, \coh_x^G)
$$

The simplest \emph{common goal} functions are \emph{product} $\prod$
and \emph{mean} $\frac{\sum_n}{n}$.


\red{The $\Gamma$ function ensures \ref{eq:coh-fun-independ} coherence property,
that permits to avoid \dots}
